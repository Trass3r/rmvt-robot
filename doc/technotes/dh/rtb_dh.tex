\documentclass[11pt]{article}
\usepackage{geometry}                % See geometry.pdf to learn the layout options. There are lots.
\geometry{letterpaper}                   % ... or a4paper or a5paper or ... 
%\geometry{landscape}                % Activate for for rotated page geometry
%\usepackage[parfill]{parskip}    % Activate to begin paragraphs with an empty line rather than an indent
\usepackage{graphicx}
\usepackage{amssymb}
\usepackage{epstopdf}
\DeclareGraphicsRule{.tif}{png}{.png}{`convert #1 `dirname #1`/`basename #1 .tif`.png}
\usepackage[colorlinks=true,urlcolor=blue]{hyperref}
%\usepackage[table,dvipsnames]{xcolor}
\usepackage[usenames,dvipsnames,svgnames,table]{xcolor}
\usepackage{array}
\usepackage{subcaption}

\usepackage{fancyhdr}
\pagestyle{fancy}
\cfoot{\thepage}
\rfoot{Copyright \copyright\ Peter Corke 2014}
\lhead{}

\input{rvc-notation}
\setlength{\parindent}{0cm}
\usepackage{mathtools}

\numberwithin{equation}{section}


\title{Denavit-Hartenberg notation for common robots}
\author{Peter Corke}
\date{March 2014}                                           % Activate to display a given date or no date

\begin{document}
\maketitle

\section{Introduction}

Denavit-Hartenberg parameters remain one of the most confusing topics for those new to the study of robotics.
This note discusses some common robot configurations and the meaning of their various Denavit-Hartenberg parameters.
Consistent diagrams and tables of Denavit-Hartenberg parameters are used to illustrate the main points.

Fundamentally we wish to describe the pose of each link in the chain relative to the pose of the preceding link.
We would expect this to comprise \textbf{six parameters}, one of which is the joint variable --- the parameter of the joint that connects the two links.
However the Denavit-Hartenberg formalism\cite[Ch.~7]{Corke11a} has only \textbf{four parameters} to describe the spatial relationship between successive link
coordinate frames, and this is achieved by introducing two constraints\cite[p.~78]{Spong06} to the placement of those frames:
\begin{enumerate}
\item The axis $x_j$ is perpendicular to the axis $z_{j-1}$.
\item The axis $x_j$ intersects the axis $z_{j-1}$.
\end{enumerate}
The result is that link frames are sometimes constrained to be placed in locations that seem non-obvious, perhaps not even on the physical link
itself.
The choices of coordinates frames are also not unique, different people will derive different, but correct, coordinate frame assignments\cite{Spong06}.  These variants will however always lead to the same expression for the pose of the end-effector with respect to the base.

In robot kinematics it is common to partition the joints into two sets
\begin{equation}
\mat{T}_E = \mat{T}_p(q_1 \cdots q_3) \mat{T}_o(q_4 \cdots q_6) \label{eq:1}
\end{equation}
The first transform, a function of the  first three joints, controls the pose of the coordinate frame $\cframe{P}$ and its responsible for setting the position
of the end-effector frame $\cframe{E}$.
The second transform, a function of the last three joints, controls the orientation of the end-effector frame $\cframe{E}$ with respect to frame
$\cframe{P}$.  This transform is a pure rotation, and on modern robots is implemented by  a \textbf{spherical wrist} mechanism that provides an arbitrary orientation with zero translation --- the origin of the frames
$\cframe{P}$ and $\cframe{E}$ are coincident 
This  approach allows us to decouple position and orientation of the end-effector $\cframe{E}$. 

The orientation of frame $\cframe{P}$ is a function of the first three joints, but we use those degrees of freedom to control position, not orientation.  Thus the orientation of $\cframe{P}$ is a consequence of the position we are trying to achieve.  The job of the wrist it provide a rotation from
$\cframe{P}$ to $\cframe{E}$ as required by the task.  The wrist will be discussed in more
detail in Section \ref{sec:wrist}.


\subsection*{Notation}
The coincident origin of the frames
$\cframe{P}$ and $\cframe{E}$ are  indicated by a large black dot in the figures that follow.  This is the centre of the wrist and the position
of the end-effector.
The x-, y- and z-axes of the base and $\cframe{P}$ are denoted by the colored
arrows that are red, blue and green respectively.  Joint rotation axes are denoted by dashed lines.

For tables of Denavit-Hartenberg parameters rows are colored blue if they contribute to wrist position, and peach if they
contribute to wrist orientation.  White cells are related to base or tool transforms.

\tableofcontents
\pagebreak
%=================
\section{RRR arm with no shoulder offset}

	\begin{figure}[h]
	\centering
	\begin{subfigure}[b]{0.6\textwidth}
		\includegraphics[width=\textwidth]{rob1}
		\caption{Plan and side elevation}
		\label{fig:rob1}
	\end{subfigure}
	\begin{subfigure}[b]{0.3\textwidth}
		\begin{tabular}{|c|c|c|c|c|} \hline
			$j$ & $\bf \theta_j$ & $\bf d_j$ & $\bf a_j$ & $\bf \alpha_j$ \\ \hline
			\rowcolor{SkyBlue}1 &  $q_1$ & \cellcolor{White} 0 &  0& $-\pi/2$ \\
			\rowcolor{SkyBlue}2 &  $q_2$ & 0& $l_1$& 0 \\
			\rowcolor{SkyBlue}3 & $q_3$ & 0 &$l_2$  & $\pi/2$ \\
			\rowcolor{Peach}4 &  $q_4$ & 0 & 0 & $-\pi/2$ \\
			\rowcolor{Peach}5&  $q_5$ & 0 & 0 & $\pi/2$ \\
			\cellcolor{Peach}6 &  \cellcolor{Peach} $q_6$ &0  &0  & 0 \\ \hline
		\end{tabular}
		\caption{Denavit-Hartenberg parameters}
		\label{fig:dh1}
	\end{subfigure}

	\caption{Simple anthropomorphic robot with no shoulder offset.  }
	\end{figure}

Almost all robot arms are anthropomorphic, human like, and they have:
\begin{enumerate}
\item a waist rotation around a vertical axis, 
\item a shoulder rotation about a horizontal axis, 
\item an elbow rotation about a horizontal axis,
\item and a 3-axis wrist joint at the end.
\end{enumerate}

A plan and side elevation of such a robot, not including the wrist, is shown in Figure \ref{fig:rob1} with the joint angles $q_1=q_2=q_3 =0$.
The coordinate
frame directions of the base and the endpoint are shown as colored arrows.

The Denavit-Hartenberg parameters of this robot are given in Figure \ref{fig:dh1} where the blue color is associated with the first three joints (wrist position), and the peach color is associated with the last three joints (the wrist).
The joint variables $q_1 \cdots q_6$ appear in the first column, the $\theta$ column, since this robot has all revolute joints.

This robot has only two significant dimensions, the lengths of the upper and lower arm which are denoted $l_1$ and $l_2$ respectively.  They appear in the $a$ column.  All other $a_j$ and $d_j$ values are zero.

The pose of $\cframe{P}$ is given by
\[
\mat{T}_p = \mat{R}_z(q_1) \mat{R}_x(-\frac{\pi}{2}) \mat{R}_z(q_2)  \mat{T}_x(l_1) \mat{R}_z(q_3) \mat{T}_x(l_2) \mat{R}_x(\frac{\pi}{2})
\]
From inspection of Figure \ref{fig:rob1} we could directly write\cite{Corke07a} a  simpler sequence of transforms
\[
\mat{T}_p = \mat{R}_z(q_1)y \mat{R}_y(q_2)  \mat{T}_x(l_1) \mat{R}_y(q_3) \mat{T}_x(l_2) 
\]
which uses only 5, rather than 7, elementary transforms --- the extra complexity in the first sequence is due to the constraints introduced by the 
Denavit-Hartenberg formalism.

\subsection{Base transform}
The origin of the robot's coordinate system is at the intersection of joints 1 and 2, a point typically inside the shoulder mechanism.
The value of $d_1$ is typically set to zero, but most real robots are mounted on some kind of pedestal above the floor.
This parameter could be used to set the height of the $q_2$ axis at some arbitrary height $h$ above the
ground plane, as shown in Figure  \ref{fig:rob1}.

If the robot is at some arbitrary pose with respect to a world coordinate system then we can rewrite \eq{eq:1} as
\[
\mat{T}_E = \mat{T}_b \mat{T}_p(q_1 \cdots q_3) \mat{T}_o(q_4 \cdots q_6) 
\]
where $\mat{T}_b$ is the robot's base transform.

\pagebreak
%=================
\section{Arms with a shoulder offset}
\subsection{RRR arm general case}

	\begin{figure}[h]
	\centering
	\begin{subfigure}[b]{0.6\textwidth}
		\includegraphics[width=\textwidth]{rob2}
		\caption{Plan and side elevation}
		\label{fig:rob2}
	\end{subfigure}
	\begin{subfigure}[b]{0.3\textwidth}
		\begin{tabular}{|c|c|c|c|c|} \hline
			$j$ & $\bf \theta_j$ & $\bf d_j$ & $\bf a_j$ & $\bf \alpha_j$ \\ \hline
			\rowcolor{SkyBlue}1 &  $q_1$ & \cellcolor{White} 0 &  $o_1$& $-\pi/2$ \\
			\rowcolor{SkyBlue}2 & $q_2$ & $o_2$ & $l_1$& 0 \\
			\rowcolor{SkyBlue}3 &  $q_3$ & $o_3$ &$l_2$  & $\pi/2$ \\
			\rowcolor{Peach}4 & $q_4$ & 0 & 0 & $-\pi/2$ \\
			\rowcolor{Peach}5 &  $q_5$ & 0 & 0 & $\pi/2$ \\
			\cellcolor{Peach} 6 & \cellcolor{Peach} $q_6$ & 0 &0  & 0 \\ \hline
		\end{tabular}
		\caption{Denavit-Hartenberg parameters}
		\label{fig:dh2}
	\end{subfigure}
	\caption{Simple anthropomorphic robot with a shoulder offset.}
	\end{figure}


Many robots have the arm assembly horizontally to one side of the base x-axis, as shown in Figure \ref{fig:rob2}.  
Examples include the PUMA 560 and most of
the Fanuc and ABB arms.
Such robots have a left and right-handed working configuration\cite[Sec~7.4.4]{Corke11a} --- the configuration shown in Figure \ref{fig:rob2} is left handed.  However by
changing joints 1 and 2 by $180\deg$ the arm will be in a right-handed configuration.

This robot has two significant dimensions, the lengths of the upper and lower arm which are denoted $l_1$ and $l_2$ respectively as before.
It also has three offsets denoted $o_1$, $o_2$ and $o_2$ respectively.  
Either or both of $o_2$ and $o_2$ will cause the
end frame to be displaced in the $\cframe{P}$ y-direction.  One introduces the shift at the shoulder, the other at the elbow.
The choice of how the offset is partitioned depends on where you choose to physically place the coordinate frames --- it makes no difference to the end-point position.

Offset $o_1$ is less common and is typically zero.  As shown in the diagram it implies that the rotational axes of joints 1 and 2 do not 
intersect.

The pose of $\cframe{P}$ is given by
\[
\mat{T}_p = \mat{R}_z(q_1) \mat{T}_x(o_1) \mat{R}_x(-\frac{\pi}{2}) \mat{R}_z(q_2)  \mat{T}_z(o_2) \mat{T}_x(l_1) \mat{R}_z(q_3) \mat{T}_z(o_3) \mat{T}_x(l_2) \mat{R}_x(\frac{\pi}{2})
\]

\pagebreak


%=================
\subsection{RRR PUMA 560 robot special case}\label{sec:puma560}

	\begin{figure}[h]
	\centering
	\begin{subfigure}[b]{0.6\textwidth}
		\centering
		\includegraphics[width=0.8\textwidth]{rob3}
		\caption{Plan and side elevation}
	\label{fig:rob3} 
	\end{subfigure}
	\begin{subfigure}[b]{0.3\textwidth}
		\begin{tabular}{|c|c|c|c|c|} \hline
			$j$ & $\bf \theta_j$ & $\bf d_j$ & $\bf a_j$ & $\bf \alpha_j$ \\ \hline
			\rowcolor{SkyBlue}1 &  $q_1$ & \cellcolor{White} 0 &  0 & $+\pi/2$ \\
			\rowcolor{SkyBlue}2 &  $q_2$ & $o_1$ & $l_1$& 0 \\
			\rowcolor{SkyBlue}3 & $q_3$ & $o_2$ &$o_3$  & $-\pi/2$ \\
			\rowcolor{Peach}4 &  $q_4$ & \cellcolor{SkyBlue} $l_2$  & 0 & $\pi/2$ \\
			\rowcolor{Peach}5 &$q_5$ & 0 & 0 & $-\pi/2$ \\
			\cellcolor{Peach} 6 & \cellcolor{Peach} $q_6$ &0  &0  & 0 \\ \hline
		\end{tabular}
		\caption{Denavit-Hartenberg parameters}
		\label{fig:dh3}
	\end{subfigure}
	\caption{PUMA 560 anthropomorphic robot with a shoulder offset.}
	\end{figure}

The classical PUMA 560 robot is a special case of the above, but there are some important differences:
\begin{itemize}
\item $\alpha_1$ and $\alpha_3$ have opposite signs to the models discussed above, and this changes the direction of the
rotational axes for joints 2 and 3, the dashed arrows are reversed.
\item the robot has a very clear shoulder offset and in the Denavit-Harteberg parameters it can be expressed using the
$o_1$ and $o_2$ offsets described above.   However in most models $o_1=0$ and the offset is lumped into the elbow joint rather than the shoulder.
\item there is an oddity in the mechanical design, an elbow offset --- the centre lines of the upper and lower arms do not intersect.
In the side elevation this is shown in exagerated fashion by the offset $o_3$.  This offset is described by $a_3$ which we used previously to express the lower arm length $l_2$.

At this point we've run out of parameters in the first three joints, so we need to pull a rather unintuitive trick.
We consider the upper arm pointing vertically for the case where $q_3=0$, and 
 we use one of the joint 4 (wrist) parameters $d_4$ to represent the lower arm length $l_2$.
 
 The parameters of the PUMA 560 robot are shown in Figure \ref{fig:dh3} and the color coding now clearly shows that the wrist position and orientation is not clearly split between the first and last
three joints.
 \end{itemize}

The pose of $\cframe{P}$ is more complex in this case, we must take into account the first \textbf{four} joints
\[
\mat[0]{T}_4 = \mat{R}_z(q_1) \mat{R}_x(\frac{\pi}{2}) \mat{R}_z(q_2)  \mat{T}_z(o_1) \mat{T}_x(l_1) \mat{R}_z(q_3) \mat{T}_z(o_2) \mat{T}_x(o_3) \mat{R}_x(-\frac{\pi}{2})  \mat{R}_z(q_4)  \underbracket{\mat{T}_z(l_2)} \mat{R}_x(-\frac{\pi}{2}) 
\]
The indicated term for lower-arm length term can be moved one place to the left since translation along z is invariant to rotation about z
\[
\mat[0]{T}_4 = \mat{R}_z(q_1) \mat{R}_x(\frac{\pi}{2}) \mat{R}_z(q_2)  \mat{T}_z(o_1) \mat{T}_x(l_1) \mat{R}_z(q_3) \mat{T}_z(o_2) \mat{T}_x(o_3) \mat{R}_x(-\frac{\pi}{2})   \mat{T}_z(l_2)  \underbracket{\mat{R}_z(q_4) \mat{R}_x(-\frac{\pi}{2})}
\]
The last two terms affect only orientation, not position, allowing us to write a transform expression for the position of $\cframe{P}$
\[
\mat{T}_p = \mat{R}_z(q_1) \mat{R}_x(\frac{\pi}{2}) \mat{R}_z(q_2)  \mat{T}_z(o_1) \mat{T}_x(l_1) \mat{R}_z(q_3) \mat{T}_z(o_2) \mat{T}_x(o_3) \mat{R}_x(-\frac{\pi}{2})   \mat{T}_z(l_2)
\]

\pagebreak

%=================
\subsection{RRP arm with prismatic joint}

	\begin{figure}[h]
	\centering
	\begin{subfigure}[b]{0.6\textwidth}
	\centering
		\includegraphics[width=0.4\textwidth]{rob4}
		\caption{Plan and side elevation}
	\label{fig:rob4} 
	\end{subfigure}
	\begin{subfigure}[b]{0.3\textwidth}
		\begin{tabular}{|c|c|c|c|c|} \hline
			$j$ & $\bf \theta_j$ & $\bf d_j$ & $\bf a_j$ & $\bf \alpha_j$ \\ \hline
			\rowcolor{SkyBlue}1 &  $q_1$ & \cellcolor{White} 0 &  0 & $-\pi/2$ \\
			\rowcolor{SkyBlue}2 &  $q_2$ & $o_1$ & 0& $+\pi/2$ \\
			\rowcolor{SkyBlue}3 & $-\pi/2$ & $q_3$ & 0  & 0 \\
			\rowcolor{Peach}4 &  $q_4$ & 0  & 0 & $-\pi/2$  \\
			\rowcolor{Peach}5 &$q_5$ & 0 & 0 & $+\pi/2$ \\
			\cellcolor{Peach} 6 & \cellcolor{Peach} $q_6$ & 0 & 0 & 0 \\ \hline
		\end{tabular}
		\caption{Denavit-Hartenberg parameters}
		\label{fig:dh4}
	\end{subfigure}
	\caption{Stanford  robot with a shoulder offset and prismatic joint.}
	\end{figure}

The Stanford arm is a not really an anthropomorphic arm.  It has a waist and a shoulder like the other robots that we have discussed, but the
length of the arm is variable --- it has a  sliding, or prismatic, third joint.  There is also shoulder offset which is
handled by the offset parameter $o_1$.

The Denavit-Hartenberg formalism requires that the prismatic extension must be in the z-direction, it is the $d_3$ parameter so in this case
$d_3 = q_3$.
Therefore we are required to draw the arm in a vertical configuration.
There is a, somewhat arbitrary, rotation $\theta_3 = -\pi/2$, which makes the y-axis of $\cframe{P}$ point to the right --- in all the other robot models
we consider it is the x-axis that points to the right.

The pose of $\cframe{P}$ is given by
\[
\mat{T}_p = \mat{R}_z(q_1) \mat{R}_x(-\frac{\pi}{2}) \mat{R}_z(q_2)  \mat{T}_z(o_1)  \mat{R}_z(-\frac{\pi}{2}) \mat{T}_z(q_3)
\]

%=================
\section{Spherical wrist}\label{sec:wrist}
Recapping from the introduction, it is common in robot kinematics  to partition the joints into two sets as given by \eq{eq:1}
\[
\mat{T}_E = \mat{T}_p(q_1 \cdots q_3) \mat{T}_o(q_4 \cdots q_6) 
\]
The first transform, a function of the  first three joints, controls the pose of the coordinate frame $\cframe{P}$ and its responsible for setting the position
of the end-effector frame $\cframe{E}$.
The second transform, a function of the last three joints, controls the orientation of the end-effector frame $\cframe{E}$ with respect to frame
$\cframe{P}$.  This transform is a pure rotation, and on modern robots is implemented by  a \textbf{spherical wrist} mechanism that provides an arbitrary orientation with zero translation --- the origin of the frames
$\cframe{P}$ and $\cframe{E}$ are coincident 
This  approach allows us to decouple position and orientation of the end-effector $\cframe{E}$. 

The orientation of frame $\cframe{P}$ is a function of the first three joints, but we use those degrees of freedom to control position, not orientation.  Thus the orientation of $\cframe{P}$ is a consequence of the position we are trying to achieve.  The job of the wrist it provide a rotation from
$\cframe{P}$ to $\cframe{E}$ as required by the task.  

The robot's wrist assembly is attached to to frame $\cframe{P}$, indicated by the black dot on the diagrams above.
The wrist mechanism is typically a three degree
of freedom mechanism (the last three joints of the robot) which allows an arbitrary orientation to be achieved.

The common Denavit-Hartenberg parameters a spherical wrist are shown in peach color
\begin{center}
\begin{tabular}{|c|c|c|c|c|} \hline
$j$ & $\bf \theta_j$ & $\bf d_j$ & $\bf a_j$ & $\bf \alpha_j$ \\ \hline
\rowcolor{Peach}4 & $q_4$ & 0 & 0 & $-\pi/2$ \\
\rowcolor{Peach}5 &  $q_5$ & 0 & 0 & $\pi/2$ \\
\cellcolor{Peach} 6 & \cellcolor{Peach} $q_6$ & 0 &0  & 0 \\ \hline
\end{tabular}
\end{center}

Critical aspects of a spherical wrist are:
\begin{itemize}
\item $\alpha_4$ and $\alpha_5$ have magnitudes of $\pi/2$ but opposite signs.  For some kinematic models they are $-\pi/2, +\pi/2$ respectively, 
in others they are $+\pi/2, -\pi/2$ respectively.
\item All the length parameters $d_j$ and $a_j$ are zero, to ensure pure rotation.

The PUMA 560 is a special case.  Because the z-axis of the $\cframe{3}$ frame points into the wrist mechanism, we can rearrange the
transform string so that the translation due to $d_4$ comes before the $q_4$ rotation (see Section \ref{sec:puma560}).  In this case, and this case only, $d_4$ can be
non-zero.
\end{itemize}

For the case  $\alpha_4=-\pi/2$ and $\alpha_5=+\pi/2$ we can write a string of elementary transformations
\[
\mat{T}_w = \mat{R}_z(q_4) \underbracket{\mat{R}_x(-\frac{\pi}{2})  \mat{R}_z(q_5)  \mat{R}_x(\frac{\pi}{2})}  \mat{R}_z(q_6)
\]
which is a sequence of pure rotations, so the overall transform is a pure rotation --- there is no translation.
The indicated term  can be simplified as
\[
\mat{R}_x(-\frac{\pi}{2})  \mat{R}_z(q_5)  \mat{R}_x(\frac{\pi}{2}) = \mat{R_y}(q_5)
\]
which gives the wrist rotation as
\[
\mat{T}_w = \mat{R}_z(q_4) \mat{R_y}(q_5)  \mat{R}_z(q_6)
\]
which is the ZYZ Euler angle sequence.  The wrist mechanism is a physical instantiation of the Euler angle sequence.

For the case  $\alpha_4=+\pi/2$ and $\alpha_5=-\pi/2$ we can write a string of elementary transformations
\begin{align}
\mat{T}_w &= \mat{R}_z(q_4) \underbracket{\mat{R}_x(\frac{\pi}{2})  \mat{R}_z(q_5)  \mat{R}_x(-\frac{\pi}{2})}_{\mat{R_y}(q_5)}  \mat{R}_z(q_6) \\
&= \mat{R}_z(q_4) \mat{R_y}(-q_5)  \mat{R}_z(q_6)
\end{align}
which is also a ZYZ Euler angle sequence but with the middle angle negated.

\subsection{Tool transform}
The coordinate frame of the end-effector $\cframe{E}$  is in the centre of the wrist assembly --- physically inside the mechanism.  Typically a robot carries a tool, the end of which
is at some fixed pose with respect to the  frame $\cframe{E}$.  

The tip of the tool is often referred to as the Tool Centre Point (TCP).
There are two ways in which the tool offset can be specified.
\begin{enumerate}
\item
There are three parameters in the last row of the Denavit-Hartenberg parameter table that we have previously shown in white.
If we assign parameters to them
\begin{center}
\begin{tabular}{|c|c|c|c|c|} \hline
$j$ & $\bf \theta_j$ & $\bf d_j$ & $\bf a_j$ & $\bf \alpha_j$ \\ \hline
\rowcolor{Peach}4 &  $q_4$ & 0  & 0 & $-\pi/2$ \\
\rowcolor{Peach}5 &$q_5$ & 0 & 0 & $\pi/2$ \\
\cellcolor{Peach} 6 & \cellcolor{Peach} $q_6$ & $z_t$ & $x_t$ & $\alpha_t$  \\ \hline
\end{tabular}
\end{center}
then we can write a transform
\[
\mat{T}(x_t, 0, z_t) \mat{R}_x(\alpha_t)
\]
which may be sufficiently general for many tasks.

\item We can append an arbitrary tool transform $\mat{T}_t$ to the robot's pose, rewriting  \eq{eq:1} as
\[
\mat{T}_E = \mat{T}_p(q_1 \cdots q_3) \mat{T}_o(q_4 \cdots q_6)  \mat{T_t}
\]
This approach might be considered for one of two reasons.  Firstly, the tool transform possible using the joint 6 Denavit-Hartenberg parameters might be insufficient, for instance, we may require an offset in the tool y-direction.  Secondly, we may not wish to modify the Denavit-Hartenberg parameters, preferring to keep them as a \textit{pure} robot model.
\end{enumerate}

\bibliographystyle{ieeetr}
\bibliography{strings,book,publist}



\end{document}  