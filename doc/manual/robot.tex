\documentclass[a4paper]{report}

\usepackage{times}
\usepackage{graphicx}
\usepackage{wrapfig}
\usepackage{pdfpages}
\usepackage{hyperref}
\usepackage{fancyhdr}

\hypersetup{
  colorlinks=true,
  linkcolor=black
  }
  
\def\topfraction{.9}
\def\bottomfraction{.9}
\def\textfraction{.1}
\def\floatpagefraction{.9}

\setlength{\parindent}{0mm}
\usepackage{parskip}
\usepackage{color}
\usepackage{sectsty}
\allsectionsfont{\sffamily}
\makeatletter
\newcommand\funcsection{%
\@startsection{section}{1}{\z@}%
  {-3.5ex \@plus -1ex \@minus -.2ex}%
  {2.3ex \@plus.2ex}%
  {\color{red}\sffamily\huge\bfseries}}
\makeatother

\input{release.tex}

%\usepackage{multind}
\usepackage{multicol}

% printindex stuff

\makeatletter
\def\printindex#1#2{\subsection*{#2}
  \@input{#1.ind}}
  
\def\theindex{\parindent\z@
\parskip\z@ plus .3pt\relax\let\item\@idxitem}
\def\@idxitem{\par\hangindent 40pt}
\def\subitem{\par\hangindent 40pt \hspace*{20pt}}
\def\subsubitem{\par\hangindent 40pt \hspace*{30pt}}
\def\endtheindex{}
\def\indexspace{\par \vskip 10pt plus 5pt minus 3pt\relax}
\makeatother

\usepackage{fancyvrb}
\fvset{formatcom=\color{blue},fontseries=c,fontfamily=courier,xleftmargin=4mm,commentchar=!}
\DefineVerbatimEnvironment{Code}{Verbatim}{formatcom=\color{blue},fontseries=c,fontfamily=courier,fontsize=\footnotesize,xleftmargin=4mm,commentchar=!}

\pagestyle{empty}

\def\Mlab{MATLAB$^{\textsuperscript{\textregistered}}$}

\begin{document}
% include the graphical front page
\includepdf{titlepage}
\thispagestyle{empty}
\newpage
% inside front cover
\vspace*{\fill}
\begin{tabular}{ll}
Release & \release \\
Release date & \reldate \\[20pt]
Licence & LGPL \\
Toolbox home page &  \url{http://www.petercorke.com/robot} \\
Discussion group & \url{http://groups.google.com.au/group/robotics-tool-box}
\end{tabular}
\vspace*{\fill}
\hrule
Copyright \textcopyright 2015 Peter Corke\\
peter.i.corke$@$gmail.com\\
\url{http://www.petercorke.com}
\newpage
\vspace*{\fill}
\setlength{\fboxsep}{10pt}%



\pagestyle{headings}        % Gives page headings at top of page
\lfoot{Robotics Toolbox \release\ for \Mlab}
\rfoot{Copyright \copyright Peter Corke 2015}

\newpage
\setcounter{section}{0}
\addcontentsline{toc}{section}{Preface}
\cleardoublepage
\chapter*{Preface}
\pagestyle{fancyplain}
\begin{wrapfigure}{l}{4cm}
\vspace{-2ex}\includegraphics[width=3.5cm]{frontcover.pdf}
\end{wrapfigure}
This, the ninth major release of the Toolbox, represents twenty years of %% CHANGE
development and a substantial level of maturity.
This version captures a large number of changes and extensions generated over the last two years
which support my new book ``\textit{Robotics, Vision \& Control}'' shown to the left.

The Toolbox has always provided many functions that are useful for the study and simulation
of classical arm-type robotics, for example
such things as kinematics, dynamics, and  trajectory generation.
The Toolbox is based on a very general method of representing the kinematics
and dynamics of serial-link manipulators.
These parameters  are encapsulated in \Mlab\  objects ---  robot objects
can be created by the user for any serial-link manipulator and a number
of examples are provided for well know robots such as the Puma 560 and the
Stanford arm amongst others.
The Toolbox also provides functions for manipulating and converting
between datatypes such
as vectors, homogeneous transformations and unit-quaternions which are necessary
to represent 3-dimensional position and orientation.

This ninth release of the Toolbox has been significantly extended to support mobile robots.
For ground robots the Toolbox includes standard path planning algorithms (bug, distance transform, D*, PRM), kinodynamic planning (RRT),
localization (EKF, particle filter), map building (EKF) and simultaneous localization and mapping (EKF), and
a Simulink model a of non-holonomic vehicle.
The Toolbox also includes a detailed Simulink model for a quadrotor flying robot.

The routines are generally written in a straightforward manner which allows
for easy understanding, perhaps at the expense of computational efficiency.
If you feel strongly about computational efficiency then you can always
rewrite the function to be more efficient,
compile the M-file using the \Mlab\   compiler, or
create a MEX version.

This manual is now essentially auto-generated from the comments in the \Mlab\ code itself which reduces the effort
in maintaining code and a separate manual as I used to --- the downside is that there are no worked examples and figures in the manual.
However the book ``\textit{Robotics, Vision \& Control}''  provides a detailed discussion (600 pages, nearly 400 figures and 1000 code examples)
of how to use the Toolbox functions to
solve many types of problems in robotics.


\newpage
\tableofcontents
\newpage
\chapter*{Functions by category}
\addcontentsline{toc}{section}{Functions by category}
\begin{multicols}{2}
\input{funcidx_body.tex}
\end{multicols}
\newpage
\chapter{Introduction}
\section{What's changed}

\subsection{New features and changes to RTB 9.10}
Features of this dot point release include:
\begin{itemize}
\item A major pass over all the documentation working on clarity and consistency.
\item \texttt{startup\_rvc} now checks whether an update to RTB is available.
\item Fixed the bug in \texttt{SerialLink.plot} for prismatic joints and modified DH parameters where the links were missing or in the wrong place.
\item New methods for \texttt{SerialLink}
  \begin{itemize}
   \item \texttt{edit} which presents all the parameters as a table in a figure window and allows
interactive editing.
   \item \texttt{fellipse} and \texttt{vellipse} which respectively plot the force and velocity ellipsoid of the
robot.
\item  \texttt{trchain}  which describes forward kinematics as a minimal series of elementary transforms.
\item  numerical inverse kinematic: \texttt{ikcon} and \texttt{ikunc} for constrained and unconstrained joint angles, based on the  MATLAB Optimisation Toolbox.  These provide a good alternative to the existing method 
\texttt{ikine}.  Contributed by Bryan Moutrie.
\item  \texttt{pay} and \texttt{paycap}  to determine the effect of payload and maximum payload capability.  Contributed by Bryan Moutrie.
\item   \texttt{collisions} which interfaces to the public domain package pHRIWARE (by Bryan Moutrie) to perform collision checking between a robot arm and static and moving objects which are described by simple 3D shape primitives.
 \end{itemize}
\item A new function called \texttt{models} which lists all the robot models and their keywords.  Allows searching by keywords.  Becoming useful
as the number of model files increases.
\item Prototype models for Baxter, NAO and Kuka KR5 robots.
\item New version of \texttt{rtbdemo} that uses a proper GUI.
\item Update of the V-REP interface to support version 3.1.x and a demo created, see \texttt{rtbdemo}.
\item An increasing number of functions now have a \texttt{'deg'} option which allows it to accept angle input in units of degrees rather than the default of
radians.
%\item New path  planning modules \texttt{Astar}, \texttt{AstarPO} and \texttt{AstarMOO} for classical, pareto-optimal and multi-objective respectively versions of the A* algorithm.  Also  \texttt{DstarPO} and \texttt{DstarPO} for respectively pareto-optimal and multi-objective versions of the D* algorithm.  Contributed by Alex Lavin.
\item New Simulink blocks to support N-rotor flyers, eg. hexa- and octo-rotors.  This includes a new control mixer block and a generalized
N-rotor dynamics block.  Graphics has been updated to render the appropriate number of rotors.  The vehicle model structure must now include
an element \texttt{nrotors} to specify the number of rotors.
\item \texttt{roblocks}, the Toolbox Simulink block library is now a \texttt{.slx} file, rather than a \texttt{.mdl} file and all models have been updated
to suit.   Some older Simulink models had atrophied and have been updated.

\end{itemize}

\subsection{Earlier changes to RTB 9}

\begin{figure}[b]
\centering
\includegraphics[width=6cm]{plot3d.pdf}
\caption{New rendered robot model using the \texttt{plot3d()} method.}\label{fig:plot3d}
\end{figure}

\begin{itemize}
\item A major rewrite of \texttt{CodeGenerator}
\item A major rewrite of \texttt{ikine6s} to handle a number of specific cases: robot with no shoulder offset, robot with
shoulder offset (can have lefty/right configuration), Stanford arm (prismatic third joint), Puma 560 arm.  The previous code made lots
of assumptions applicable to the Puma, which caused errors for other 6-axis robots with spherical wrists.
\item Symbolic inverse kinematics (developmental) can be found for robots with 2, 3 or 6 DOF.  See \texttt{SerialLink.ikine\_sym}.
\item Aesthetic updates to \texttt{plot()} and \texttt{teach()} methods of the \texttt{SerialLink} object

\item A new method \texttt{plot3d()} which uses STL format solid models to render realistic looking robots as shown in Figure
\ref{fig:plot3d}.  This requires STL models,
such as those shipped with the package ARTE by Arturo Gil (\url{https://arvc.umh.es/arte}).

\item Subclasses of \texttt{Link} called \texttt{Revolute}, \texttt{Prismatic}, \texttt{RevoluteMDH} and \texttt{PrismaticMDH} that can be used to make your
code clearer and more concise.
\item The behaviour of Fast RNE is a bit different.  The MATLAB version \texttt{@SerialLink/rne.m} is always executed, and it makes
the decision whether or not to invoke the MEX file.  The MEX file is executed if:
	\begin{enumerate}
	\item the robot is not symbolic, and
	\item the \texttt{SerialLink} property \texttt{fast} is true (ie. the `nofast' option is not given), and
	\item the MEX file exists.
	\end{enumerate}
	
\item A significant number of new robot models.
\item A major renovation of \texttt{DHFactor} to bring it up to spec with the lastest version of Java.
\item A 'flip' option added to \texttt{tr2eul}.
\item A new method \texttt{A} for \texttt{SerialLink} object that computes a sequence of \texttt{Link} A matrices.
\item A new method \texttt{trchain} for \texttt{SerialLink} object that emits the kinematic model as a sequence of elementary
transforms.
\item A new method \texttt{trchain} that expresses kinematics as a chain of elementary transforms.
\item A bunch of functions with suffix \texttt{2} that deal with SE(2) and SO(2) transforms.
\item An improved version of the demo \texttt{rtbdemo}, with more functions and an improved interface.  It uses the common
function \texttt{runscript} to step through the individual demo scripts.  It works even better with \texttt{cprintf} from MATLAB Central.
\item A set of classes (experimental) to interface with the V-REP robotics simulation engine by Coppelia Robotics.  See
\texttt{robot/interfaces/VREP}.
\item Since 9.8 the Toolbox now contains the Robotic Symbolic Toolbox
by J\"{o}rn Malzahn.  There are additional functions, as well as symbolic
support throughout the SerialLink class.
\item Many bug fixes
\end{itemize}

\section{Migrating from RTB 8 and earlier}
If you're upgrading from RTB 8 or earlier note a significant number of changes summarised below.
\begin{itemize}
\item The command \texttt{startup\_rvc} should be executed before using the Toolbox.  This sets up the MATLAB search paths correctly.
\item The Robot class is now named SerialLink to be more specific.
\item Almost all functions that operate on a SerialLink object are now methods rather than functions, for example plot() or fkine(). In practice this makes little difference to the user but operations can now be expressed as robot.plot(q) or plot(robot, q). Toolbox documentation now prefers the former convention which is more aligned with object-oriented practice.
\item The parametrers to the Link object constructor are now in the order: theta, d, a, alpha. Why this order? It's the order in which the link transform is created: RZ(theta) TZ(d) TX(a) RX(alpha).
\item All robot models now begin with the prefix mdl\_, so puma560 is now mdl\_puma560.
\item The function drivebot is now the SerialLink method teach.
\item The function ikine560 is now the SerialLink method ikine6s to indicate that it works for any 6-axis robot with a spherical wrist.
\item The link class is now named Link to adhere to the convention that all classes begin with a capital letter.
\item The \texttt{robot} class is now called \texttt{SerialLink}.  It is created from a vector of \texttt{Link} objects, not a cell array.
\item The quaternion class is now named Quaternion to adhere to the convention that all classes begin with a capital letter.
\item A number of utility functions have been moved into the a directory common since they are not robot specific.
\item skew no longer accepts a skew symmetric matrix as an argument and returns a 3-vector, this functionality is provided by the new function vex.
\item tr2diff and diff2tr are now called tr2delta and delta2tr
\item ctraj with a scalar argument now spaces the points according to a trapezoidal velocity profile (see lspb). To obtain even spacing provide a uniformly spaced vector as the third argument, eg. linspace(0, 1, N).
\item The RPY functions tr2rpy and rpy2tr assume that the roll, pitch, yaw rotations are about the X, Y, Z axes which is consistent with common conventions for vehicles (planes, ships, ground vehicles). For some applications (eg. cameras) it useful to consider the rotations about the Z, Y, X axes, and this behaviour can be obtained by using the option 'zyx' with these functions (note this is the pre release 8 behaviour).
\item Many functions now accept MATLAB style arguments given as trailing strings, or string-value pairs.  These are parsed by the internal
function \texttt{tb\_optparse}.
\end{itemize}


\subsection{New functions}
Release 9 introduces considerable new functionality, in particular for mobile robot control, navigation and localization:
\begin{itemize}
\item Mobile robotics:
\begin{description}
\item[Vehicle] Model of a mobile robot that has the ``bicycle" kinematic model (car-like). For given inputs it updates the robot state and returns noise corrupted odometry measurements. This can be used in conjunction with a ``driver" class such as RandomPath which drives the vehicle between random waypoints within a specified rectangular region.
\item[Sensor]
\item[RangeBearingSensor] Model of a laser scanner RangeBearingSensor, subclass of Sensor, that works in conjunction with a Map object to return range and bearing to invariant point features in the environment.
\item[EKF] Extended Kalman filter EKF can be used to perform localization by dead reckoning or map featuers, map buildings and simultaneous localization and mapping.
\item[DXForm] Path planning classes: distance transform DXform, D* lattice planner Dstar, probabilistic roadmap planner PRM, and rapidly exploring random tree RRT.
\item Monte Carlo estimator ParticleFilter.
 \end{description}
 
\item Arm robotics: \texttt{jsingu}, \texttt{qplot}, \texttt{DHFactor} (a simple means to generate the Denavit-Hartenberg kinematic model of a robot from a sequence of elementary transforms)
 
\item Trajectory related: \texttt{lspb}, \texttt{tpoly}, \texttt{mtraj}, \texttt{mstraj}
   
\item General transformation: \texttt{homtrans}, \texttt{se2}, \texttt{se3}, \texttt{wtrans}, \texttt{vex} (performs the inverse function to skew, it converts a skew-symmetric matrix to a 3-vector)
 
\item Data structures:
  \begin{description}
\item \texttt{Pgraph} represents a non-directed embedded graph, supports plotting and minimum cost path finding.
\item \texttt{Polygon} a generic 2D polygon class that supports plotting, intersectio/union/difference of polygons, line/polygon intersection, point/polygon containment.
 \end{description}
 
\item Graphical functions:
  \begin{description}
 \item[\texttt{trprint}] compact display of a transform in various formats.
\item[\texttt{trplot}] display a coordinate frame in SE(3)
\item[\texttt{trplot2}] as above but for SE(2)
\item[\texttt{tranimate}] animate the motion of a coordinate frame
\item[\texttt{plot\_box}] plot a box given TL/BR corners or center+WH, with options for edge color, fill color and transparency.
\item[\texttt{plot\_circle}] plot one or more circles, with options for edge color, fill color and transparency.
\item[\texttt{plot\_sphere}] plot a sphere, with options for edge color, fill color and transparency.
\item[\texttt{plot\_ellipse}] plot an ellipse, with options for edge color, fill color and transparency.
\item[\texttt{plot\_ellipsoid}] plot an ellipsoid, with options for edge color, fill color and transparency.
\item[\texttt{plot\_poly}] plot a polygon, with options for edge color, fill color and transparency.
 \end{description}
 
 \item Utility:
  \begin{description}
\item[about] display a one line summary of a matrix or class, a compact version of whos
\item[tb\_optparse] general argument handler and options parser, used internally in many functions.
\end{description}
\item Lots of Simulink models are provided in the subdirectory \texttt{simulink}.  These models all have the prefix \texttt{sl\_}.
\end{itemize}

\subsection{General improvements}
\begin{itemize}
\item Many functions now accept MATLAB style arguments given as trailing strings, or string-value pairs.  These are parsed by the internal
function \texttt{tb\_optparse}.
\item Many functions now handle sequences of rotation matrices or homogeneous transformations.
\item Improved error messages in many functions
\item Removed trailing commas from if and for statements
\end{itemize}

\section{How to obtain the Toolbox}
The Robotics Toolbox is freely available from the Toolbox home
page at 
\begin{quote}
\url{http://www.petercorke.com}
\end{quote}

The web page requests some information from you
regarding such as your country, type of organization and application.
This is just a means for me to gauge interest and to remind myself
that this is a worthwhile activity.

The file is available in zip format (.zip).  Download it and unzip it.
Files all unpack to the correct parts of a hiearchy of directories (folders)
headed by \texttt{rvctools}.

If you already have the Machine Vision Toolbox installed then download
the zip file to the directory above the existing \texttt{rvctools} directory,
and then unzip it.
The files from this zip archive will properly interleave with the Machine
Vision Toolbox files.

Ensure that the folder \texttt{rvctools} is on your \Mlab\ search
path.  You can do this by issuing the \texttt{addpath} command at 
the \Mlab\ prompt.
Then issue the command \texttt{startup\_rvc} and it will add a number
of paths to your \Mlab\ search path.
You need to setup the path every time you start \Mlab\ but you can 
automate this by setting up environment variables, editing your 
\texttt{startup.m} script, or by pressing the ``Update Toolbox Path
Cache" button under \Mlab\ General preferences.

A menu-driven demonstration can be invoked by the function {\tt rtbdemo}.

\subsection{Documentation}
This document {\tt robot.pdf} is a manual that describes all functions in the Toolbox.
It is auto-generated from the comments in the \Mlab\ code and is fully hyperlinked:
to external web sites, the table of content to functions, and the ``See also'' functions
to each other.

The same documentation is available online in
alphabetical order at \url{http://www.petercorke.com/RTB/r9/html/index_alpha.html}
or by category at \url{http://www.petercorke.com/RTB/r9/html/index.html}.
Documentation is also available via the \Mlab\ help browser, ``Robotics
Toolbox" appears under the Contents.


\section{MATLAB version issues}
The Toolbox has been tested under R2014b.  Compatibility problems are increasingly likely the older your version of \Mlab\ is.

\section{Use in teaching}
This is definitely encouraged!
You are free to put the PDF manual (\texttt{robot.pdf} or the web-based documentation {\texttt{html/*.html} on a server for class
use.
If you plan to distribute paper copies of the PDF manual then every copy must include the first two pages (cover and licence).

\section{Use in research}
If the Toolbox helps you in your endeavours then I'd appreciate you citing the Toolbox when you publish.
The details are:
\begin{verbatim}
@book{Corke11a,
    Author = {Peter I. Corke},
    Date-Added = {2011-01-12 08:19:32 +1000},
    Date-Modified = {2012-07-29 20:07:27 +1000},
    Note = {ISBN 978-3-642-20143-1},
    Publisher = {Springer},
    Title = {Robotics, Vision \& Control: Fundamental Algorithms in {MATLAB}},
    Year = {2011}}
\end{verbatim}
or
\begin{quote}
P.I. Corke, Robotics, Vision \& Control: Fundamental Algorithms in MATLAB. Springer, 2011. ISBN 978-3-642-20143-1.
\end{quote}
which is also given in electronic form in the CITATION file.

\section{Support}
There is no support!  This software is made freely available in the hope that you find it useful in solving whatever problems
you have to hand.
I am happy to correspond with people who have found genuine
bugs or deficiencies but my response time can be long and I can't guarantee that I respond to your email.

\textbf{I can guarantee that I will not respond to any requests for help with assignments or homework, no matter
how urgent or important they might be to you.  That's what your teachers, tutors, lecturers and professors are paid to do.}

You might instead like to communicate with other users via 
the Google Group called ``Robotics and Machine Vision Toolbox'' 
\begin{quote}
\url{http://groups.google.com.au/group/robotics-tool-box}
\end{quote}
which is a forum for discussion.
You need to signup in order to post, and the signup process is moderated by me so allow a few
days for this to happen.  I need you to write a few words about why you want to join the list
so I can distinguish you from a spammer or a web-bot.

\section{Related software}

\subsection{Octave}
Octave is an open-source mathematical environment that is very similar to \Mlab, but it has some important differences particularly with respect
to graphics and classes.
Many Toolbox functions work just fine under Octave.
Three important classes (Quaternion, Link and SerialLink) will not work so modified versions of these classes is provided in the subdirectory
called \texttt{Octave}.  Copy all the directories from \texttt{Octave} to the main Robotics Toolbox directory.

The Octave port is a second priority for support and upgrades and is offered in the hope that you find it useful. 

\subsection{Python version}
A python implementation of the Toolbox at \url{http://code.google.com/p/robotics-toolbox-python}.
All core functionality of the release 8 Toolbox is present including kinematics, dynamics, Jacobians,
quaternions etc.  It is based on the python numpy class.  The main current limitation is the lack of good 3D graphics support
but people are working on this.  Nevertheless this version of the toolbox is very usable and of course you don't need a \Mlab\ licence
to use it.  Watch this space.

\subsection{Machine Vision toolbox}
Machine Vision toolbox (MVTB) for {\Mlab}.  This was described in an article
\begin{verbatim}
@article{Corke05d,
        Author = {P.I. Corke},
        Journal = {IEEE Robotics and Automation Magazine},
        Month = nov,
        Number = {4},
        Pages = {16-25},
        Title = {Machine Vision Toolbox},
        Volume = {12},
        Year = {2005}}
\end{verbatim}
and provides a very wide range of useful computer vision functions beyond the Mathwork's Image Processing
Toolbox.  You can obtain this from \url{http://www.petercorke.com/vision}.

\section{Contributing to the Toolboxes}
I am very happy to accept contributions for inclusion in future versions of the
toolbox.  You will, of course, be suitably acknowledged (see below).

\section{Acknowledgements}
I have corresponded with a great many people via email since the first 
release of this Toolbox.  Some have identified bugs and shortcomings
in the documentation, and even better, some have provided bug fixes and
even new modules, thankyou.  See the file \texttt{CONTRIB} for details.

J\"{o}rn Malzahn has donated a considerable amount of code, his
Robot Symbolic Toolbox for MATLAB.
Bryan Moutrie has contributed parts of his open-source package phiWARE to RTB, the remainder of that package can be found online.
%Alexander Lavin has contributed variants of A* and D* planners.
Other mentions to Gautam Sinha, Wynand Smart for models of industrial robot arm, Paul Pounds for the quadrotor and related models, 
and Paul Newman (Oxford) for inspiring the mobile robot code.


\renewcommand{\section}{\funcsection}
%\setcounter{secnumdepth}{-1}
%\settocdepth{section}
\newpage
\chapter{Functions and classes}
\input{all}

\bibliographystyle{ieeetr}
\bibliography{strings,robot,control,dynamics,kinematics,force,grind,publist,software}
\end{document}
