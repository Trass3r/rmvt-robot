%\documentstyle[titlepage,tut,twoside,longtable,fancybox]{article}
\documentclass{article}
%\documentstyle[titlepage,ref,twoside]{article}
\usepackage{verbatim}
\usepackage{fancyheadings,fancybox,longtable}
\usepackage{times}

\def\topfraction{.9}
\def\bottomfraction{.9}
\def\textfraction{.1}
\def\floatpagefraction{.9}

\setlength{\topmargin}{10.4mm}      % top of header to top of paper
\setlength{\textheight}{220mm}      % height of body
\setlength{\textwidth}{128mm}  % width of text
\setlength{\evensidemargin}{36.6mm} % even page - left paper edge to text
\setlength{\oddsidemargin}{36.6mm}  % odd page - left paper edge to text 
\setlength{\headsep}{12mm} %distance bet. header & body text
\setlength{\footskip}{12mm}%distance bet. body of text & baseline of footer
\setlength{\parindent}{0pt} %don't indent paragraphs
\setlength{\parskip}{2mm} %distance between paragraphs

\setlength{\hfuzz}{10pt} %allow for wider text

% Include font file: either NFSS or PostScript:
%% Fonts defined with long PostScript names 

\newfont{\Rfon}{NewCenturySchlbk-Roman} % Main text font
\newfont{\School}{NewCenturySchlbk-Roman}
\newfont{\si}{NewCenturySchlbk-Italic}
\newfont{\ital}{NewCenturySchlbk-Italic}
\newfont{\emph}{NewCenturySchlbk-Italic}
\newfont{\bs}{NewCenturySchlbk-Bold}

% Courier if you have it, otherwise
% substitute Courier: 
\newfont{\vtt}{Courier at 9pt}
\newfont{\Cour}{Courier at 10pt}
\newfont{\Mono}{Courier at 10pt}
\newfont{\Cfon}{Courier at 9pt}  % Computer font
\newfont{\TTfon}{Courier at 9pt}  % Typewriter font for matlab command display
\newfont{\Lfon}{Courier at 7pt}
\newfont{\Zfon}{Courier at 7pt}

\newfont{\Mfon}{Helvetica-Bold at 14pt}
\newfont{\Refon}{Helvetica-Bold at 12pt} % Font for ref pgs headings
\newfont{\Ifon}{NewCenturySchlbk-Bold scaled \magstep2}
\newfont{\Tfon}{NewCenturySchlbk-Bold scaled \magstep4}
\newfont{\Nfon}{NewCenturySchlbk-Bold at 35pt}
\newfont{\NNfon}{Times-Bold at 32pt}
\newfont{\NNNfon}{Times-Bold at 40pt}
\newfont{\NNNNfon}{NewCenturySchlbk-Bold at 40pt}
\newfont{\rbig}{Times-Bold at 27pt} %vphantom in underlining section heads
\newfont{\tms}{Times-Bold at 18pt} %for section heads
\newfont{\tmss}{Helvetica-Bold at 14pt} %for subsection heads
\newfont{\tmsss}{Helvetica-Bold at 12pt} %for subsubsection heads
\newfont{\tmssss}{Helvetica-BoldOblique at 10pt} %for subsubsubsection heads
\newfont{\nineTR}{NewCenturySchlbk-Roman at 9pt}
\newfont{\nineTI}{NewCenturySchlbk-Italic at 9pt}
\newfont{\elevenTI}{NewCenturySchlbk-Italic at 11pt}
\newfont{\bt}{NewCenturySchlbk-Bold at 9pt}
\newfont{\eightTR}{NewCenturySchlbk-Roman at 8pt}
\newfont{\sevenTR}{NewCenturySchlbk-Roman at 7pt}
\newfont{\eightTI}{NewCenturySchlbk-Italic at 8pt}
\newfont{\Heading}{NewCenturySchlbk-Roman at 70pt}
\newfont{\Headingi}{NewCenturySchlbk-Italic at 60pt}
\newfont{\Afon}{NewCenturySchlbk-Roman at 18pt}

\def\und{{\vtt\char"5F}}


% Fonts defined in NFSS 

%\newfont{\Rfon}{pncr} % Main text font
\newfont{\School}{pncr}
\newfont{\si}{pncri}
%\newfont{\emph}{pncri}
\newfont{\ital}{pncri}
\newfont{\bs}{pncb}

% MathWorks uses Monospace--this
% NFSS definition is for Courier. 
\newfont{\Cour}{pcrr at 10pt}
\newfont{\Mono}{pcrr at 10pt}
\newfont{\vtt}{pcrr at 9pt}
\newfont{\Cfon}{pcrr at 9pt}  % Computer font
\newfont{\TTfon}{pcrr at 9pt}  % Typewriter font for matlab command display
\newfont{\Lfon}{pcrr at 7pt}
\newfont{\Zfon}{pcrr at 7pt}

\newfont{\Mfon}{phvb at 14pt}
\newfont{\Refon}{phvb at 12pt} % Font for ref pgs headings
\newfont{\Ifon}{pncb scaled \magstep2}
\newfont{\Tfon}{pncb scaled \magstep4}
\newfont{\Nfon}{pncb at 35pt}
\newfont{\NNfon}{ptmb at 32pt}
\newfont{\NNNfon}{ptmb at 40pt}
\newfont{\NNNNfon}{pncb at 40pt}
\newfont{\rbig}{ptmb at 27pt} %just for vphantom in underlining section heads
\newfont{\tms}{ptmb at 18pt} %for section heads
\newfont{\tmss}{phvb at 14pt} %for subsection heads
\newfont{\tmsss}{phvb at 12pt} %for subsubsection heads
\newfont{\tmssss}{phvbo at 10pt} %for subsubsubsection heads
\newfont{\nineTR}{pncr at 9pt}
\newfont{\nineTI}{pncri at 9pt}
\newfont{\elevenTI}{pncri at 11pt}
\newfont{\bt}{pncb at 9pt}
\newfont{\eightTR}{pncr at 8pt}
\newfont{\sevenTR}{pncr at 7pt}
\newfont{\eightTI}{pncri at 8pt}
%\newfont{\Heading}{pncb at 70pt}
%\newfont{\Headingi}{pncri at 60pt}
\newfont{\Heading}{ptmb at 70pt}
\newfont{\Headingi}{ptmri at 60pt}
%\newfont{\Afon}{pncr at 18pt}
\newfont{\Afon}{ptmr at 18pt}

\def\und{{\vtt\char"5F}}

% Psfig/TeX Release 1.2
% dvips version
%
% All software, documentation, and related files in this distribution of
% psfig/tex are Copyright 1987, 1988 Trevor J. Darrell
%
% Permission is granted for use and non-profit distribution of psfig/tex 
% providing that this notice be clearly maintained, but the right to
% distribute any portion of psfig/tex for profit or as part of any commercial
% product is specifically reserved for the author.
%
% $Header: /home/pic/cvsroot/robot-toolbox/doc/psfig.tex,v 1.1 2002-06-07 11:19:05 pic Exp $
% $Source: /home/pic/cvsroot/robot-toolbox/doc/psfig.tex,v $
%
% Thanks to Greg Hager (GDH) and Ned Batchelder for their contributions
% to this project.
%
\catcode`\@=11\relax
\newwrite\@unused
\def\typeout#1{{\let\protect\string\immediate\write\@unused{#1}}}
\typeout{psfig/tex 1.2-dvips}


%% Here's how you define your figure path.  Should be set up with null
%% default and a user useable definition.

\def\figurepath{./}
\def\psfigurepath#1{\edef\figurepath{#1}}

%
% @psdo control structure -- similar to Latex @for.
% I redefined these with different names so that psfig can
% be used with TeX as well as LaTeX, and so that it will not 
% be vunerable to future changes in LaTeX's internal
% control structure,
%
\def\@nnil{\@nil}
\def\@empty{}
\def\@psdonoop#1\@@#2#3{}
\def\@psdo#1:=#2\do#3{\edef\@psdotmp{#2}\ifx\@psdotmp\@empty \else
    \expandafter\@psdoloop#2,\@nil,\@nil\@@#1{#3}\fi}
\def\@psdoloop#1,#2,#3\@@#4#5{\def#4{#1}\ifx #4\@nnil \else
       #5\def#4{#2}\ifx #4\@nnil \else#5\@ipsdoloop #3\@@#4{#5}\fi\fi}
\def\@ipsdoloop#1,#2\@@#3#4{\def#3{#1}\ifx #3\@nnil 
       \let\@nextwhile=\@psdonoop \else
      #4\relax\let\@nextwhile=\@ipsdoloop\fi\@nextwhile#2\@@#3{#4}}
\def\@tpsdo#1:=#2\do#3{\xdef\@psdotmp{#2}\ifx\@psdotmp\@empty \else
    \@tpsdoloop#2\@nil\@nil\@@#1{#3}\fi}
\def\@tpsdoloop#1#2\@@#3#4{\def#3{#1}\ifx #3\@nnil 
       \let\@nextwhile=\@psdonoop \else
      #4\relax\let\@nextwhile=\@tpsdoloop\fi\@nextwhile#2\@@#3{#4}}
% 
%
\def\psdraft{
	\def\@psdraft{0}
	%\typeout{draft level now is \@psdraft \space . }
}
\def\psfull{
	\def\@psdraft{100}
	%\typeout{draft level now is \@psdraft \space . }
}
\psfull
\newif\if@prologfile
\newif\if@postlogfile
\newif\if@noisy
\def\pssilent{
	\@noisyfalse
}
\def\psnoisy{
	\@noisytrue
}
\psnoisy
%%% These are for the option list.
%%% A specification of the form a = b maps to calling \@p@@sa{b}
\newif\if@bbllx
\newif\if@bblly
\newif\if@bburx
\newif\if@bbury
\newif\if@height
\newif\if@width
\newif\if@rheight
\newif\if@rwidth
\newif\if@clip
\newif\if@verbose
\def\@p@@sclip#1{\@cliptrue}

%%% GDH 7/26/87 -- changed so that it first looks in the local directory,
%%% then in a specified global directory for the ps file.

\def\@p@@sfile#1{\def\@p@sfile{null}%
	        \openin1=#1
		\ifeof1\closein1%
		       \openin1=\figurepath#1
			\ifeof1\typeout{Error, File #1 not found}
			\else\closein1
			    \edef\@p@sfile{\figurepath#1}%
                        \fi%
		 \else\closein1%
		       \def\@p@sfile{#1}%
		 \fi}
\def\@p@@sfigure#1{\def\@p@sfile{null}%
	        \openin1=#1
		\ifeof1\closein1%
		       \openin1=\figurepath#1
			\ifeof1\typeout{Error, File #1 not found}
			\else\closein1
			    \def\@p@sfile{\figurepath#1}%
                        \fi%
		 \else\closein1%
		       \def\@p@sfile{#1}%
		 \fi}

\def\@p@@sbbllx#1{
		%\typeout{bbllx is #1}
		\@bbllxtrue
		\dimen100=#1
		\edef\@p@sbbllx{\number\dimen100}
}
\def\@p@@sbblly#1{
		%\typeout{bblly is #1}
		\@bbllytrue
		\dimen100=#1
		\edef\@p@sbblly{\number\dimen100}
}
\def\@p@@sbburx#1{
		%\typeout{bburx is #1}
		\@bburxtrue
		\dimen100=#1
		\edef\@p@sbburx{\number\dimen100}
}
\def\@p@@sbbury#1{
		%\typeout{bbury is #1}
		\@bburytrue
		\dimen100=#1
		\edef\@p@sbbury{\number\dimen100}
}
\def\@p@@sheight#1{
		\@heighttrue
		\dimen100=#1
   		\edef\@p@sheight{\number\dimen100}
		%\typeout{Height is \@p@sheight}
}
\def\@p@@swidth#1{
		%\typeout{Width is #1}
		\@widthtrue
		\dimen100=#1
		\edef\@p@swidth{\number\dimen100}
}
\def\@p@@srheight#1{
		%\typeout{Reserved height is #1}
		\@rheighttrue
		\dimen100=#1
		\edef\@p@srheight{\number\dimen100}
}
\def\@p@@srwidth#1{
		%\typeout{Reserved width is #1}
		\@rwidthtrue
		\dimen100=#1
		\edef\@p@srwidth{\number\dimen100}
}
\def\@p@@ssilent#1{ 
		\@verbosefalse
}
\def\@p@@sprolog#1{\@prologfiletrue\def\@prologfileval{#1}}
\def\@p@@spostlog#1{\@postlogfiletrue\def\@postlogfileval{#1}}
\def\@cs@name#1{\csname #1\endcsname}
\def\@setparms#1=#2,{\@cs@name{@p@@s#1}{#2}}
%
% initialize the defaults (size the size of the figure)
%
\def\ps@init@parms{
		\@bbllxfalse \@bbllyfalse
		\@bburxfalse \@bburyfalse
		\@heightfalse \@widthfalse
		\@rheightfalse \@rwidthfalse
		\def\@p@sbbllx{}\def\@p@sbblly{}
		\def\@p@sbburx{}\def\@p@sbbury{}
		\def\@p@sheight{}\def\@p@swidth{}
		\def\@p@srheight{}\def\@p@srwidth{}
		\def\@p@sfile{}
		\def\@p@scost{10}
		\def\@sc{}
		\@prologfilefalse
		\@postlogfilefalse
		\@clipfalse
		\if@noisy
			\@verbosetrue
		\else
			\@verbosefalse
		\fi
}
%
% Go through the options setting things up.
%
\def\parse@ps@parms#1{
	 	\@psdo\@psfiga:=#1\do
		   {\expandafter\@setparms\@psfiga,}}
%
% Compute bb height and width
%
\newif\ifno@bb
\newif\ifnot@eof
\newread\ps@stream
\def\bb@missing{
	\if@verbose{
		\typeout{psfig: searching \@p@sfile \space  for bounding box}
	}\fi
	\openin\ps@stream=\@p@sfile
	\no@bbtrue
	\not@eoftrue
	\catcode`\%=12
	\loop
		\read\ps@stream to \line@in
		\global\toks200=\expandafter{\line@in}
		\ifeof\ps@stream \not@eoffalse \fi
		%\typeout{ looking at :: \the\toks200 }
		\@bbtest{\toks200}
		\if@bbmatch\not@eoffalse\expandafter\bb@cull\the\toks200\fi
	\ifnot@eof \repeat
	\catcode`\%=14
}	
\catcode`\%=12
\newif\if@bbmatch
\def\@bbtest#1{\expandafter\@a@\the#1%%BoundingBox:\@bbtest\@a@}
\long\def\@a@#1%%BoundingBox:#2#3\@a@{\ifx\@bbtest#2\@bbmatchfalse\else\@bbmatchtrue\fi}
\long\def\bb@cull#1 #2 #3 #4 #5 {
	\dimen100=#2 bp\edef\@p@sbbllx{\number\dimen100}
	\dimen100=#3 bp\edef\@p@sbblly{\number\dimen100}
	\dimen100=#4 bp\edef\@p@sbburx{\number\dimen100}
	\dimen100=#5 bp\edef\@p@sbbury{\number\dimen100}
	\no@bbfalse
}
\catcode`\%=14
%
\def\compute@bb{
		\no@bbfalse
		\if@bbllx \else \no@bbtrue \fi
		\if@bblly \else \no@bbtrue \fi
		\if@bburx \else \no@bbtrue \fi
		\if@bbury \else \no@bbtrue \fi
		\ifno@bb \bb@missing \fi
		\ifno@bb \typeout{FATAL ERROR: no bb supplied or found}
			\no-bb-error
		\fi
		%
		\count203=\@p@sbburx
		\count204=\@p@sbbury
		\advance\count203 by -\@p@sbbllx
		\advance\count204 by -\@p@sbblly
		\edef\@bbw{\number\count203}
		\edef\@bbh{\number\count204}
		%\typeout{ bbh = \@bbh, bbw = \@bbw }
}
%
% \in@hundreds performs #1 * (#2 / #3) correct to the hundreds,
%	then leaves the result in @result
%
\def\in@hundreds#1#2#3{\count240=#2 \count241=#3
		     \count100=\count240	% 100 is first digit #2/#3
		     \divide\count100 by \count241
		     \count101=\count100
		     \multiply\count101 by \count241
		     \advance\count240 by -\count101
		     \multiply\count240 by 10
		     \count101=\count240	%101 is second digit of #2/#3
		     \divide\count101 by \count241
		     \count102=\count101
		     \multiply\count102 by \count241
		     \advance\count240 by -\count102
		     \multiply\count240 by 10
		     \count102=\count240	% 102 is the third digit
		     \divide\count102 by \count241
		     \count200=#1\count205=0
		     \count201=\count200
			\multiply\count201 by \count100
		 	\advance\count205 by \count201
		     \count201=\count200
			\divide\count201 by 10
			\multiply\count201 by \count101
			\advance\count205 by \count201
			%
		     \count201=\count200
			\divide\count201 by 100
			\multiply\count201 by \count102
			\advance\count205 by \count201
			%
		     \edef\@result{\number\count205}
}
\def\compute@wfromh{
		% computing : width = height * (bbw / bbh)
		\in@hundreds{\@p@sheight}{\@bbw}{\@bbh}
		%\typeout{ \@p@sheight * \@bbw / \@bbh, = \@result }
		\edef\@p@swidth{\@result}
		%\typeout{w from h: width is \@p@swidth}
}
\def\compute@hfromw{
		% computing : height = width * (bbh / bbw)
		\in@hundreds{\@p@swidth}{\@bbh}{\@bbw}
		%\typeout{ \@p@swidth * \@bbh / \@bbw = \@result }
		\edef\@p@sheight{\@result}
		%\typeout{h from w : height is \@p@sheight}
}
\def\compute@handw{
		\if@height 
			\if@width
			\else
				\compute@wfromh
			\fi
		\else 
			\if@width
				\compute@hfromw
			\else
				\edef\@p@sheight{\@bbh}
				\edef\@p@swidth{\@bbw}
			\fi
		\fi
}
\def\compute@resv{
		\if@rheight \else \edef\@p@srheight{\@p@sheight} \fi
		\if@rwidth \else \edef\@p@srwidth{\@p@swidth} \fi
}
%		
% Compute any missing values
\def\compute@sizes{
	\compute@bb
	\compute@handw
	\compute@resv
}
%
% \psfig
% usage : \psfig{file=, height=, width=, bbllx=, bblly=, bburx=, bbury=,
%			rheight=, rwidth=, clip=}
%
% "clip=" is a switch and takes no value, but the `=' must be present.
\def\psfig#1{\vbox {
	% do a zero width hard space so that a single
	% \psfig in a centering enviornment will behave nicely
	%{\setbox0=\hbox{\ }\ \hskip-\wd0}
	%
	\ps@init@parms
	\parse@ps@parms{#1}
	\compute@sizes
	%
	\ifnum\@p@scost<\@psdraft{
		\if@verbose{
			\typeout{psfig: including \@p@sfile \space }
		}\fi
		%
		\special{ps::[begin] 	\@p@swidth \space \@p@sheight \space
				\@p@sbbllx \space \@p@sbblly \space
				\@p@sbburx \space \@p@sbbury \space
				startTexFig \space }
		\if@clip{
			\if@verbose{
				\typeout{(clip)}
			}\fi
			\special{ps:: doclip \space }
		}\fi
		\if@prologfile
		    \special{ps: plotfile \@prologfileval \space } \fi
		\special{ps: plotfile \@p@sfile \space }
		\if@postlogfile
		    \special{ps: plotfile \@postlogfileval \space } \fi
		\special{ps::[end] endTexFig \space }
		% Create the vbox to reserve the space for the figure
		\vbox to \@p@srheight true sp{
			\hbox to \@p@srwidth true sp{
				\hss
			}
		\vss
		}
	}\else{
		% draft figure, just reserve the space and print the
		% path name.
		\vbox to \@p@srheight true sp{
		\vss
			\hbox to \@p@srwidth true sp{
				\hss
				\if@verbose{
					\@p@sfile
				}\fi
				\hss
			}
		\vss
		}
	}\fi
}}
\def\psglobal{\typeout{psfig: PSGLOBAL is OBSOLETE; use psprint -m instead}}
\catcode`\@=12\relax


\newcommand{\DEF}[1]{{\it#1}}
\newlength{\figwidth}
\newlength{\rlwidth}            % for root-locus plots
\setlength{\figwidth}{10cm}
\setlength{\rlwidth}{10cm}
\newlength{\bigwidth}           % for root-locus plots
\setlength{\bigwidth}{10cm}
\newcommand{\var}[1]{{\vtt #1}}
\newcommand{\eq}[1]{(\ref{#1})}
\newcommand{\floor}{{\rm floor}}
\newcommand{\ceil}{{\rm ceil}}
\newcommand{\ord}{{\rm Ord}}
\newcommand{\fnumber}{{\it f}-number}
\newcommand{\etal}{{et al.}}
\newcommand{\apriori}{{\it a priori}}
\renewcommand{\vec}[1]{\underline{#1}}
\newcommand{\SNR}{\mbox{SNR}}
%
% use as \sci{5}{-2} for formatting 5e-2
%
\newcommand{\sci}[2]{\mbox{$#1 \times 10^{#2}$}}
 
 %
 % macros to display units in math or LR mode, 20\unit{lx/m^2}
 %
 \newcommand{\unit}[1]{\mbox{$\rm \,#1$}}
 \newcommand{\Hz}{\unit{Hz}}
 \newcommand{\ms}{\unit{ms}}
 \newcommand{\us}{\unit{\mu s}}
 \renewcommand{\deg}{${}^\circ$}


\newcommand{\namelistlabel}[1]{\mbox{#1}\hfil}
\newenvironment{namelist}[1]{%
\begin{list}{}
{
\let\makelabel\namelistlabel
\settowidth{\labelwidth}{#1}
\setlength{\leftmargin}{1.1\labelwidth}
}
}{%
\end{list}}

\pagestyle{empty}
\begin{document}
\thispagestyle{empty}
%
% this is the HEADING  ************************************
%
%
\vspace*{\fill}
{\Headingi Robotics}

\vspace{2mm}
{\Heading TOOLBOX\\}

\vspace{5mm}
{\Afon for MATLAB}

\vspace*{2cm}
{\Afon (Release 7)}

\vspace*{\fill}

\begin{tabular}{ll}
\psfig{figure=figs/plot.eps,width=6cm} &\psfig{figure=figs/surfl.eps,width=6cm}
\end{tabular}

\vspace*{\fill}
\hrule
\parbox{50mm}{{\Afon Peter I. Corke}\\
Peter.Corke@csiro.au} \hfill {\Afon April 2002}\\
http://www.cat.csiro.au/cmst/staff/pic/robot
\newpage
\vspace*{\fill}
\setlength{\fboxsep}{10pt}%
\doublebox{%
\begin{Bflushleft}[b]
Peter I. Corke\\
CSIRO\\
Manufacturing Science and Technology\\
Pullenvale, AUSTRALIA, 4069.\\
2002\\
http://www.cat.csiro.au/cmst/staff/pic/robot
\end{Bflushleft}}
\vspace*{\fill}

\copyright 2002 by Peter I. Corke.
\noindent
\cleardoublepage

\newcommand{\Mlab}{M\eightTR ATLAB}
\newcommand{\under}[1]{\underline{\rule[-.70ex]{0cm}{4mm}#1}}


%{\School

\pagestyle{headings}        % Gives page headings at top of page


\newcommand{\be}{\begin{equation}}
\newcommand{\ee}{\end{equation}}
\newcommand{\bea}{\begin{eqnarray}}
\newcommand{\eea}{\end{eqnarray}}
\newcommand{\bi}{\begin{itemize}}
\newcommand{\ei}{\end{itemize}}
\newcommand{\ii}{\item}


\renewcommand{\baselinestretch}{1.2}        % Expands inter-line spacing

\newpage
\rightline{\NNNfon 1}
\vskip 4mm
\rightline{\NNfon Preface}
\vskip 2mm
\moveleft 1.25in\vbox{\makebox[6.25in][l]
{\vrule height .01 in width 6.25 true in}}
\markboth{Introduction}{{}}
\setcounter{section}{0}
\addcontentsline{toc}{section}{Introduction}
\vskip 5mm

\section{Introduction}
This, the seventh release of the Toolbox, represents nearly a decade of
tinkering and a substantial level of maturity.  It finally includes many of
the features I've been planning to add for some time, particularly MEX files,
Simulink support and modified Denavit-Hartenberg support.
The previous release has had thousands of downloads and the mailing list 
has hundreds of subscribers.

The Toolbox provides many functions that are useful in robotics
including such things as kinematics, dynamics, and  trajectory generation.
The Toolbox is useful for simulation as well as analyzing results from
experiments with real robots.

The Toolbox is based on a very general method of representing the kinematics
and dynamics of serial-link manipulators.
These parameters  are encapsulated in Matlab objects.  Robot objects
can be created by the user for any serial-link manipulator and a number
of examples are provided for well know robots such as the Puma 560 and the
Stanford arm.
The Toolbox also provides functions for manipulating and converting
between datatypes such
as vectors, homogeneous transformations and unit-quaternions which are necessary
to represent 3-dimensional position and orientation.

The routines are written in a straightforward manner which allows
for easy understanding, perhaps at the expense of computational efficiency.
My guide in all of this work has been the book of Paul\cite{Paul81a}, now
out of print, but which I grew up with.
If you feel strongly about computational efficiency then you can always
rewrite the function to be more efficient,
compile the M-file using the Matlab compiler, or
create a MEX version.

\subsection{What's new}
This release has some significant new functionality as well as some
bug fixes.
\begin{itemize}
\item	Full support for modified (Craig's) Denavit-Hartenberg notation,
	forward and inverse kinematics, Jacobians and forward and inverse
	dynamics.
\item	Simulink blockset library and demonstrations included, see Section
	\ref{sec:simulink}
\item	MEX implementation of recursive Newton-Euler algorithm written in
	portable C.  Speed improvements of at least 1000.  Tested on Solaris,
	Linux and Windows. See Section \ref{sec:mex}.
\item	Fixed still more bugs and missing files in quaternion code.
\item	Remove `@' notation from \var{fdyn} to allow operation under
	Matlab 5 and 6.
\item	Fairly major update of documentation to ensure consistency between
	code, online help and this manual.
\end{itemize}

All code is now under CVS control which should eliminate many of the
versioning problems I had previously due to developing the code across 
multiple computers.

\subsection{Contact}
The Toolbox home page is at
\begin{quote}
http://www.cat.csiro.au/cmst/staff/pic/robot
\end{quote}
This page will always list the current released version number as well as
bug fixes and new code in between major releases. 

A Mailing List is also available, subscriptions details are available
off that web page.

\subsection{How to obtain the Toolbox}
The Robotics Toolbox is freely available from the Toolbox home
page at 
\begin{quote}
http://www.cat.csiro.au/cmst/staff/pic/robot
\end{quote}

The files are available in either gzipped tar format (.gz) or zip
format (.zip).  The web page requests some information from you
regarding such as your country, type of organization and application.
This is just a means for me to gauge interest and to help convince my
bosses (and myself) that this is a worthwhile activity.
%BUG
%the MathWorks FTP server\linebreak
%{\tt ftp.mathworks.com} in the directory {\tt pub/contrib/misc/robot}.
%It is best to download all files in that directory since the Toolbox
%functions are quite interdependent.

The file {\tt robot.pdf} is a comprehensive manual with a tutorial introduction
and details of each Toolbox function.
A menu-driven demonstration can be invoked by the function {\tt rtdemo}.

\subsection{MATLAB version issues}
The Toolbox works with {\Mlab} version 6 and greater and has been tested 
on a Sun with version 6.

The Toolbox does not function under {\Mlab} v3.x or v4.x since those versions
do not support objects.  An older version of the Toolbox, available from
the Matlab4 ftp site is workable but lacks some features of this current
Toolbox release.


\subsection{Acknowledgements}
I am grateful for the support of my employer, CSIRO, for supporting me in
this activity and providing me with the Matlab tools and web server.

I have corresponded with a great many people via email since the first 
release of this Toolbox.  Some have identified bugs and shortcomings
in the documentation, and even better, some have provided bug fixes and
even new modules, thankyou.

\subsection{Support, use in teaching, bug fixes, etc.}
I'm always happy to correspond with people who have found genuine
bugs or deficiencies
in the Toolbox, or who have suggestions about ways to improve its
functionality.  However I draw the line at providing help for people
with their assignments and homework!

Many people are using the Toolbox for teaching and this is something that
I would encourage.  If you plan to duplicate the documentation for class
use then every copy must include the front page.

If you want to cite the Toolbox please use
\begin{verbatim}
@ARTICLE{Corke96b,
        AUTHOR             = {P.I. Corke},
        JOURNAL            = {IEEE Robotics and Automation Magazine},
        MONTH              = mar,
        NUMBER             = {1},
        PAGES              = {24-32},
        TITLE              = {A Robotics Toolbox for {MATLAB}},
        VOLUME             = {3},
        YEAR               = {1996}
}
\end{verbatim}
which is also given in electronic form in the README file.

\subsection{A note on kinematic conventions}
Many people are not aware that there are two quite different forms of 
Denavit-Hartenberg representation for serial-link manipulator kinematics:
\begin{enumerate}
\item Classical as per the original 1955 paper of Denavit and Hartenberg,
and used in textbooks such as by Paul\cite{Paul81a}, Fu etal\cite{Fu87}, or 
Spong and Vidyasagar\cite{Spong89}.
\item Modified form as introduced by Craig\cite{Craig86} in his text book.
\end{enumerate}
Both notations represent a joint as 2 translations ($A$ and $D$) and 2
rotation angles 
($\alpha$ and $\theta$).  However the expressions for the link transform
matrices are quite different.  In short, you must know which kinematic
convention your Denavit-Hartenberg parameters conform to.

Unfortunately many sources in the literature do not specify this crucial
piece of information.
Most textbooks cover only one and do not even allude to the existence of the
other.
These issues are discussed further in Section \ref{sec:kine-conventions}.

The Toolbox has full support for both the classical and
modified conventions.

\subsection{Creating a new robot definition}
Let's take a simple example like the two-link planar manipulator from
Spong \& Vidyasagar\cite{Spong89} (Figure 3-6, p73) which has the following
(standard) Denavit-Hartenberg link parameters 

\begin{center}
\begin{tabular}{|c|c|c|c|c|} \hline
Link & $a_i$ & $\alpha_i$ & $d_i$ & $\theta_i$ \\ \hline
1 & 1 & 0 & 0 & $\theta_1^*$ \\
2 & 1 & 0 & 0 & $\theta_2^*$ \\ \hline
\end{tabular}
\end{center}

where we have set the link lengths to 1.  Now we can create a pair of
link objects:
\begin{verbatim}
>> L1=link([0 1 0 0 0], 'standard') 
 
 L1 = 
  
    0.000000      1.000000        0.000000        0.000000        R    (std)
     
>> L2=link([0 1 0 0 0], 'standard')
      
 L2 = 
       
    0.000000      1.000000        0.000000        0.000000        R    (std)

>> r=robot({L1 L2})                
 
r = 
 
noname (2 axis, RR)
                grav = [0.00 0.00 9.81]         standard D&H parameters
  alpha           A               theta           D             R/P
0.000000       1.000000        0.000000        0.000000        R    (std)
0.000000       1.000000        0.000000        0.000000        R    (std)

>>
\end{verbatim}
The first few lines create link objects, one per robot link.
Note the second argument to \var{link} which specifies that the standard
D\&H conventions are to be used (this is actually the default).
The arguments to the link object can be found from
\begin{verbatim}
>> help link
    .
    .
        LINK([alpha A theta D sigma], CONVENTION)
    .
    .
\end{verbatim}
which shows the order in which the link parameters must be passed (which
is different to the column order of the table above).
The fifth element of the first argument, \var{sigma}, is a flag that 
indicates whether the
joint is revolute (\var{sigma} is zero) or primsmatic (\var{sigma} is non zero).

The link objects are passed as a cell array to the \var{robot()} function
which creates a robot object which is in turn passed to many of the 
other Toolbox functions.

Note that displays of link data include the kinematic convention in
brackets on the far right.  \var{(std)} for standard form, and 
\var{(mod)} for modified form.

The robot just created can be displayed graphically by
\begin{verbatim}
>> plot(r, [0 0])
\end{verbatim}
which will create the plot shown in Figure \ref{fig:simple-plot}.

\begin{figure}
\psfig{figure=figs/simple.eps,width=9cm}
\caption{Simple two-link manipulator model.}\label{fig:simple-plot}
\end{figure}

\subsection{Using MEX files}\label{sec:mex}
The Robotics Toolbox Release 7 includes portable C source code to generate
a MEX file version of the \var{rne} function.

The MEX file runs upto 500
times faster than the interpretted version \var{rne.m} and this is critical
for calculations involving forward dynamics.
The forward dynamics requires the calculation of the manipulator inertia
matrix at each integration time step.  The Toolbox uses a computationally
simple but inefficient method that requires evaluating the \var{rne}
function $n+1$ times, where $n$ is the number of robot axes.  For forward
dynamics the \var{rne} is the bottleneck.

The Toolbox stores all robot kinematic and inertial parameters in a
\var{robot} object, but accessing these parameters from a C language
MEX file is somewhat cumbersome and must be done on each call.  Therefore
the speed advantage increases with the number of rows in the \var{q},
\var{qd} and \var{qdd} matrices that are provided.  In other words
it is better to call \var{rne} with a trajectory, than for each point on
a trajectory.

To build the MEX file:
\begin{enumerate}
\item Change directory to the \var{mex} subdirectory of the Robotics Toolbox.
\item On a Unix system just type \texttt{make}.  For other platforms
follow the Mathworks guidelines.  You need to compile and link three files
with a command something like \texttt{mex frne.c ne.c vmath.c}.

\item If successful you now have a file called \texttt{frne.ext} where 
\var{ext} is the file extension and depends on the architecture
(\var{mexsol} for Solaris, \var{mexlx} for Linux).

\item From within Matlab \texttt{cd} into this same directory and run the 
test script
{\small
\begin{verbatim}
    >> cd ROBOTDIR/mex
    >> check
    ***************************************************************
    ************************ Puma 560 *****************************
    ***************************************************************
    ************************ normal case *****************************
    DH:  Fast RNE: (c) Peter Corke 2002
    Speedup is         17, worst case error is 0.000000
    MDH: Speedup is       1565, worst case error is 0.000000
    ************************ no gravity *****************************
    DH:  Speedup is       1501, worst case error is 0.000000
    MDH: Speedup is       1509, worst case error is 0.000000
    ************************ ext force *****************************
    DH:  Speedup is       1497, worst case error is 0.000000
    MDH: Speedup is        637, worst case error is 0.000000
    
    ***************************************************************
    ********************** Stanford arm ***************************
    ***************************************************************
    ************************ normal case *****************************
    DH:  Speedup is       1490, worst case error is 0.000000
    MDH: Speedup is       1519, worst case error is 0.000000
    ************************ no gravity *****************************
    DH:  Speedup is       1471, worst case error is 0.000000
    MDH: Speedup is       1450, worst case error is 0.000000
    ************************ ext force *****************************
    DH:  Speedup is        417, worst case error is 0.000000
    MDH: Speedup is       1458, worst case error is 0.000000
    >> 
\end{verbatim}
}

This will run the M-file and MEX-file versions of the \var{rne} function
for various robot models and options with various options.  For each case
it should report a speedup greater than one, and an error of zero.
The results shown above are for a Sparc Ultra 10.

\item Copy the MEX-file \texttt{frne.ext} into the Robotics Toolbox main
directory with the name \texttt{rne.ext}.  Thus all future references to
\var{rne} will now invoke the MEX-file instead of the M-file.  The first
time you run the MEX-file in any Matlab session it will print a one-line
identification message.
\end{enumerate}

\clearpage
\newpage
\rightline{\NNNfon 2}
\vskip 4mm
\rightline{\NNfon Using the Toolbox with Simulink}
\vskip 2mm
\moveleft 1.25in\vbox{\makebox[6.25in][l]
{\vrule height .01 in width 6.25 true in}}
\markboth{Introduction}{{}}
\setcounter{section}{1}
\addcontentsline{toc}{section}{Introduction}
\vskip 5mm

\section{Introduction}\label{sec:simulink}
Simulink is the block diagram editing and simulation environment for Matlab.
Until its most recent release Simulink has not been able to handle matrix
valued signals, and that has made its application to robotics somewhat 
clumsy.  This shortcoming has been rectified with Simulink Release 4.
Robot Toolbox Release 7 and higher includes a library of blocks for use
in constructing robot kinematic and dynamic models.

\begin{figure}[b]
\psfig{figure=figs/roblocks.eps,width=12cm}
\caption{The Robotics Toolbox blockset.}\label{fig:blockset}
\end{figure}
To use this new feature it is neccessary to include the Toolbox Simulink block
directory in your Matlab path:
\begin{verbatim}
>> addpath ROBOTDIR/simulink
\end{verbatim}
To bring up the block library
\begin{verbatim}
>> roblocks
\end{verbatim}
which will create a display like that shown in Figure \ref{fig:blockset}.

Users with no previous Simulink experience are advised
to read the relevant Mathworks manuals and experiment with the examples 
supplied.
Experienced Simulink users should find the use of the Robotics blocks quite
straightforward.  Generally there is a one-to-one correspondence between
Simulink blocks and Toolbox functions.
Several demonstrations have been included with the Toolbox in order to 
illustrate common topics in robot control and demonstrate Toolbox Simulink 
usage.
These could be considered as starting points for your own work, just select
the model closest to what you want and start changing it.
Details of the blocks can be found using the File/ShowBrowser option on
the block library window.

\section{Examples}
\subsection{Dynamic simulation of Puma 560 robot collapsing under gravity}
The Simulink model, \var{demo1}, is shown in Figure \ref{fig:demo1}, and 
the two blocks in this model would be
familiar to Toolbox users.
The \var{Robot} block is similar to the \var{fdyn()}
function and represents the forward dynamics of the robot, and the \var{plot}
block represents the \var{plot} function.
Note the parameters of the \var{Robot} block contain the robot object 
to be simulated and the initial joint angles.  
The \var{plot} block has one parameter which is the robot object to be 
displayed graphically and should be consistent with the robot being simulated.
Display options are taken from the \var{plotbotopt.m} file, see help
for \var{robot/plot} for details.
\begin{figure}
\psfig{figure=figs/demo1.eps,width=12cm}
\caption{Robotics Toolbox example \texttt{demo1}, Puma robot
collapsing under gravity.}\label{fig:demo1}
\end{figure}

To run this demo first create a robot object in
the workspace,typically by using the \var{puma560} command, then start
the simulation using Simulation/Start option from the model toolbar.
\begin{verbatim}
>> puma560
>> demo1
\end{verbatim}

\subsection{Dynamic simulation of a simple robot with flexible
transmission}
The Simulink model, \var{demo2}, is shown in Figure \ref{fig:demo2}, and
represents a simple 2-link robot with flexible or compliant transmission.
The first joint receives a step position
demand change at time 1s.  The resulting oscillation and dynamic coupling
between the two joints can be seen clearly.  Note that the drive model
comprises spring plus damper, and that the joint position control loops
are simply unity negative feedback.

\begin{figure}
\psfig{figure=figs/demo2.eps,width=12cm}
\caption{Robotics Toolbox example \texttt{demo2}, simple flexible
2-link manipulator.}\label{fig:demo2}
\end{figure}

To run this demo first create a 2-link robot object in
the workspace,typically by using the \var{twolink} command, then start
the simulation using Simulation/Start option from the model toolbar.
\begin{verbatim}
>> twolink
>> demo2
\end{verbatim}

\subsection{Computed torque control}
The Simulink model, \var{demo3}, shown in Figure \ref{fig:demo3}, is
for a Puma560 with a computed torque control structure.
This is a ``classical"
dynamic control technique in which the rigid-body dynamic model is inverted
to compute the demand torque for the robot based on current joint angles
and joint angle rates and demand joint angle acceleration.  This model
introduces the \var{rne} block which computes the inverse dynamics using
the recursive Newton-Euler algorithm (see \var{rne} function), and the
\var{jtraj} block which computes a vector quintic polynomial.  \var{jtraj} has
parameters which include the initial and final values of the each output
element as well as the overall motion time.  Initial and final velocity are
assumed to be zero.
\begin{figure}[b]
\psfig{figure=figs/demo3.eps,width=12cm}
\caption{Robotics Toolbox example \texttt{demo3}, computed torque
control.}\label{fig:demo3}
\end{figure}

In practice of course the dynamic model of the robot is not exactly known,
we can only invert our best estimate of the rigid-body dynamics.  In the
simulation we can model this by using the \var{perturb} function to alter
the parameters of the dynamic model used in the \var{rne} block --- note
the 'P/' prefix on the model name displayed by that block.
This means that the inverse dynamics are computed for a slightly different 
dynamic model to the robot under control and shows the effect of model
error on control performance.

To run this demo first create a robot object in
the workspace,typically by using the \var{puma560} command, then start
the simulation using Simulation/Start option from the model toolbar.
\begin{verbatim}
>> puma560
>> demo3
\end{verbatim}

\subsection{Torque feedforward control}
The Simulink model \var{demo4} demonstrates torque feedforward control,
another ``classical" dynamic control technique in which the demanded 
torque is computed using the
\var{rne} block and added to the error torque computed from position and
velocity error.  It is instructive to compare the structure of this
model with \var{demo3}.
The inverse dynamics are not in the forward path and since
the robot configuration changes relatively slowly, they can be computed at a
low rate (this is illustrated by the zero-order hold block sampling at 20Hz).

To run this demo first create a robot object in
the workspace,typically by using the \var{puma560} command, then start
the simulation using Simulation/Start option from the model toolbar.
\begin{verbatim}
>> puma560
>> demo4
\end{verbatim}

\subsection{Cartesian space control}
The Simulink model, \var{demo5}, shown in Figure \ref{fig:demo5}, demonstrates
Cartesian space motion control.
There are two conventional approaches to 
this. Firstly, resolve the Cartesian space demand to joint space using
inverse kinematics and then perform the control in joint space.  The second,
used here, is to compute the error in Cartesian space and resolve that to
joint space via the inverse Jacobian.  This eliminates the need for inverse
kinematics within the control loop, and its attendent problems of multiple
solutions.  It also illustrates some additional Simulink blocks.

\begin{figure}[b]
\psfig{figure=figs/demo5.eps,width=12cm}
\caption{Robotics Toolbox example \texttt{demo5}, Cartesian space
control.}\label{fig:demo5}
\end{figure}

This demonstration is for a Puma 560
robot moving the tool in a circle of radius 0.05m centered at the point 
$(0.5,\, 0,\, 0)$.  The difference between the Cartesian demand and the current 
Cartesian pose (in end-effector coordinates) is computed by the 
\var{tr2diff} block which produces a 
differential motion described by a 6-vector.
The Jacobian block has as its input the current manipulator joint angles and
outputs the Jacobian matrix.  Since the differential motion is with respect
to the end-effector we use the JacobianN block rather than Jacobian0.
We use standard Simulink block to invert the Jacobian and multiply it by the
differential motion.  The result, after application of a simple 
proportional gain, is the joint space motion required to 
correct the Cartesian error.
The robot is modelled by an integrator as a simple rate control device, or
velocity servo.

This example also demonstrates the use of the \var{fkine} block for forward
kinematics and the \var{T2xyz} block which extracts the translational part
of the robot's Cartesian state for plotting on the XY plane.

This demonstration is very similar to the numerical method used to solve
the inverse kinematics problem in \var{ikine}.

To run this demo first create a robot object in
the workspace,typically by using the \var{puma560} command, then start
the simulation using Simulation/Start option from the model toolbar.
\begin{verbatim}
>> puma560
>> demo5
\end{verbatim}


\subsection{Image-based visual servoing}
The Simulink model, \var{demo6}, shown in Figure \ref{fig:demo6}, demonstrates
image-based visual servoing (IBVS)\cite{Hutchinson96}.
This is quite a complex example that simulates not only the robot but also
a camera and the IBVS algorithm.
The camera is assumed to be mounted on the robot's end-effector and this
coordinate is passed into the camera block so that the relative position of
the target with respect to the camera can be computed.
Arguments to the camera block include the 3D coordinates of the target points.
The output of the camera is the 2D image plane coordinates of the target
points.
The target points are used to compute an image Jacobian matrix which is
inverted and multiplies the desired motion of the target points on the image
plane.
The desired motion is simply the difference between the observed target points
and the desired point positions.
The result is a velocity screw which drives the 
robot to the desired pose with respect to a square target.
\begin{figure}[b]
\psfig{figure=figs/demo6.eps,width=12cm}
\caption{Robotics Toolbox example \texttt{demo6}, image-based visual
servoing.}\label{fig:demo6}
\end{figure}

When the simulation starts a new window, the camera view, pops up.  We
see that initially the square target is off to one side and somewhat oblique.
The image plane errors are mapped by an image Jacobian into desired Cartesian
rates, and these are futher mapped by a manipulator Jacobian into joint
rates which are applied to the robot which is again modelled as a rate
control device.  This closed-loop system is performing a Cartesian 
positioning task using information from a camera rather than encoders
and a kinematic model (the Jacobian is a weak kinematic model).  Image-based
visual servoing schemes have been found to be extremely robust with respect to
errors in the camera model and manipulator Jacobian.

\newpage
\rightline{\NNNfon 3}
\vskip 4mm
\rightline{\NNfon Tutorial}
\vskip 2mm
\moveleft 1.25in\vbox{\makebox[6.25in][l]
{\vrule height .01 in width 6.25 true in}}
\markboth{Introduction}{{}}
\setcounter{section}{2}
\addcontentsline{toc}{section}{Introduction}
\vskip 5mm

\section{Manipulator kinematics}  \label{sec:robot-kine}
Kinematics is the study of motion without regard to the forces which cause
it.  Within kinematics one studies the position, velocity and acceleration,
and all higher order derivatives of the position variables.
The kinematics of manipulators involves the study of the geometric
and time based properties of the motion, and in particular how the various
links move with respect to one another and with time.

Typical robots are \DEF{serial-link manipulators} comprising a set of bodies, called \DEF{links},
in a chain, connected by \DEF{joints}\footnote{Parallel link and serial/parallel hybrid structures 
are possible, though much less common in industrial manipulators.}.
Each joint has one degree of freedom,
either translational or rotational.  For a manipulator with $n$ joints
numbered from $1$ to $n$,
there are $n+1$ links, numbered from $0$ to $n$.  Link $0$ is the 
base of the manipulator, generally fixed, and link $n$ carries the end-effector.
Joint $i$ connects links $i$ and $i-1$.


A link may be considered as a rigid body defining the relationship between
two neighbouring joint axes.  A link can be specified by two numbers, the
{\em link length} and {\em link twist}, which define the relative location
of the two axes in space.  The link parameters for the first and last
links are meaningless, but are arbitrarily chosen to be 0.
Joints may be described by two parameters.  The {\em link offset}
is the distance from one link to the next along the axis of the joint.
The {\em joint angle} is the rotation of one link with respect to the
next about the joint axis.

To facilitate describing the location of each link we affix a coordinate
frame to it --- frame $i$ is attached to link $i$.
Denavit and Hartenberg\cite{Hartenberg55} proposed a
matrix method
of systematically assigning coordinate systems to each link of an
articulated chain.
The axis of revolute joint $i$ is aligned with $z_{i-1}$.
The  $x_{i-1}$ axis is directed along the normal from $z_{i-1}$ to
$z_i$ and for intersecting axes is parallel to $z_{i-1} \times z_i$.
The link and joint parameters may be summarized as:

\begin{center}
\begin{tabular}{lrp{8cm}}
link length & $a_i$ & the offset distance between the $z_{i-1}$ and $z_i$ axes
along the $x_i$ axis;\\
link twist & $\alpha_i$ & the angle from the $z_{i-1}$ axis to the $z_i$ axis about the
$x_i$ axis;\\
link offset & $d_i$ & the distance from the origin of frame $i-1$ to the
$x_i$ axis along the $z_{i-1}$ axis;\\
joint angle & $\theta_i$ & the angle between the $x_{i-1}$ and
$x_i$ axes about the $z_{i-1}$ axis.
\end{tabular}
\end{center}

\begin{figure}
\centerline{\psfig{figure=figs/kine-std.id,width=\bigwidth}}
\centerline{\mbox{(a) Standard form}}
\centerline{\psfig{figure=figs/kine-mod.id,width=\bigwidth}}
\centerline{\mbox{(b) Modified form}}
\caption{Different forms of Denavit-Hartenberg notation.}
\label{fig:kinematics}
\end{figure}

For a revolute axis $\theta_i$ is the
joint variable and $d_i$ is constant, while for a prismatic joint $d_i$ is variable,
and $\theta_i$ is constant.  In many of the formulations that follow we use
generalized coordinates, $q_i$, where
\[
q_i = \left \{ \begin{array}{ll}
	\theta_i & \mbox{for a revolute joint} \\
	d_i & \mbox{for a prismatic joint}
	\end{array} \right .
\]
and generalized forces
\[
Q_i = \left \{ \begin{array}{ll}
	\tau_i & \mbox{for a revolute joint} \\
	f_i & \mbox{for a prismatic joint}
	\end{array} \right .
\]

The Denavit-Hartenberg (DH) representation results
in a 4x4 homogeneous transformation matrix
\begin{equation}
{}^{i-1} {\bf A}_i = \left [ \begin{array}{cccc}
	\cos \theta_i & -\sin \theta_i \cos \alpha_i & \sin \theta_i \sin
	\alpha_i & a_i \cos \theta_i\\
	\sin \theta_i & \cos \theta_i \cos \alpha_i & -\cos \theta_i \sin
	\alpha_i & a_i \sin \theta_i\\
	0 & \sin \alpha_i & \cos \alpha_i & d_i\\
0 & 0 & 0 & 1
	\end{array} \right ]    \label{eq:Amatrix}
\end{equation}
representing each link's
coordinate frame with respect to the previous link's coordinate system; that is
\begin{equation}
{}^0 {\bf T}_i = {}^0 {\bf T}_{i-1} \;{}^{i-1} {\bf A}_i   \label{eq:joint-xform}
\end{equation}
where ${}^0 {\bf T}_i$ is the homogeneous transformation describing the pose
of coordinate frame $i$ with respect to the world coordinate system $0$.

Two differing methodologies have been established for assigning coordinate frames,
each of which allows some freedom in the actual coordinate
frame attachment:\label{sec:kine-conventions}
\begin{enumerate}
\item Frame $i$ has its origin along the axis of joint $i+1$, as described
by Paul\cite{Paul81a} and Lee\cite{Fu87,Lee82a}.
\item Frame $i$ has its origin along the axis of joint $i$, and is frequently
referred to as `modified Denavit-Hartenberg' (MDH) form\cite{Craig89}.
This form is commonly used in literature dealing with manipulator dynamics.
The link transform matrix for this form differs from (\ref{eq:Amatrix}).
\end{enumerate}
Figure \ref{fig:kinematics} shows the notational differences between
the two forms.  Note that $a_i$ is always the length of link $i$, but is
the displacement between the origins of frame $i$ and frame $i+1$ in one
convention, and frame $i-1$ and frame $i$ in the other\footnote{
Many papers when tabulating the `modified' kinematic parameters
of manipulators list $a_{i-1}$ and $\alpha_{i-1}$ not $a_i$ and $\alpha_i$.}.
The Toolbox provides kinematic functions for both of these
conventions --- those for modified DH parameters are prefixed by `m'.

\subsection{Forward and inverse kinematics}
For an n-axis rigid-link manipulator, the \DEF{forward kinematic solution} gives
the coordinate frame, or pose, of the last link.
It is obtained
by repeated application of (\ref{eq:joint-xform})
\begin{eqnarray}
{}^0 {\bf T}_n &=& {}^0 {\bf A}_1 {}^1 {\bf A}_2 \cdots {}^{n-1}{\bf A}_n \\
    &=& {\cal K}(\vec{q}) 
\end{eqnarray}
which is the product of the coordinate frame transform matrices for each link.
The pose of the end-effector has 6 degrees of freedom in Cartesian space,
3 in translation and 3 in rotation, so robot manipulators commonly have 6
joints or degrees of freedom to allow arbitrary end-effector pose.
The overall manipulator transform ${}^0 {\bf T}_n$ is frequently written as
${\bf T}_n$, or ${\bf T}_6$ for a 6-axis robot.
The forward kinematic solution may be computed for any manipulator, irrespective 
of the number of joints or kinematic structure.

Of more use in manipulator path planning is the \DEF{inverse kinematic solution}
\begin{equation}
\vec{q} = {\cal K}^{-1} ( {\bf T})
\end{equation}
which gives the joint
angles required to reach the specified end-effector position.  In general
this solution is non-unique, and for some classes of manipulator no
closed-form solution exists.  If the manipulator has more than 6 joints it is
said to be \DEF{redundant} and the solution for joint angles is under-determined.
If no solution can be determined for a particular manipulator pose that 
configuration is said to be \DEF{singular}.  
The singularity may be due to an alignment
of axes reducing the effective degrees of freedom,
or the point ${\bf T}$ being out of reach.

The manipulator Jacobian matrix, $\bf J_\theta$, transforms velocities 
in joint space to velocities of the end-effector in Cartesian space.
For an $n$-axis manipulator the end-effector Cartesian velocity is
\bea
 {}^0 \underline{\dot{x}}_n &=& {}^0 {\bf J}_\theta  \underline{\dot{q}} \\
 {}^{t_n} \underline{\dot{x}}_n &=& {}^{t_n} {\bf J}_\theta  \underline{\dot{q}}
\eea
in base or end-effector coordinates respectively and where $\vec{x}$ is
the Cartesian velocity represented by a 6-vector.
For a 6-axis manipulator
the Jacobian is square and provided it is not singular can be inverted to solve for joint 
rates in terms of end-effector Cartesian rates.
The Jacobian will not be invertible  at a kinematic singularity, and
in practice will be poorly conditioned in the vicinity of the 
singularity, resulting in high joint rates.
A control scheme based on Cartesian rate control 
\be
\underline{\dot{q}} = {}^0 {\bf J}_\theta^{-1} \;  {}^0 \underline{\dot{x}}_n 
\ee
was proposed by Whitney\cite{Whitney72} 
and is known as \DEF{resolved rate motion control}.
For two frames $A$ and $B$ related by  ${}^A {\bf T}_B = [\vec{n}\; \vec{o}\; \vec{a}\; \vec{p}]$ the
Cartesian velocity in frame $A$ may be transformed to frame $B$ by
\be
{}^B \vec{\dot{x}} = {}^B {\bf J}_A \,{}^A \vec{\dot{x}}
\ee
where the Jacobian is
given by Paul\cite{Paul81b} as
\be
{}^B {\bf J}_A = f({}^A {\bf T}_B) = \left[ \begin{array}{cc}
	[\vec{n}\; \vec{o}\; \vec{a}]^T & [\vec{p}\times \vec{n}\; \vec{p}\times \vec{o}\; \vec{p}\times \vec{a}]^T \\
	0 & [\vec{n}\; \vec{o}\; \vec{a}]^T \end{array} \right]  \label{eq:tr2jac}
\ee


%%%%%%%%%%%%% RIGID-BODY DYNAMICS%%%%%%%%%%%%%%%
\section{Manipulator rigid-body dynamics}  \label{sec:robot-rigid-body}
Manipulator dynamics is concerned with the equations of motion, the way
in which the manipulator moves in response to torques applied by the actuators,
or external forces.
The history and mathematics of the dynamics of serial-link
manipulators is well covered by Paul\cite{Paul81a} and
Hollerbach\cite{Hollerbach82}.
There are two problems related to manipulator dynamics that are important to solve:
\begin{itemize}
\item \DEF{inverse dynamics} in which the manipulator's equations of motion  are
solved for given motion to determine the generalized forces, discussed 
further in Section \ref{sec:robot-control}, and
\item \DEF{direct dynamics} in which the equations of motion are integrated to determine
the generalized coordinate response to applied generalized forces discussed further
in Section \ref{robot:direct-dyn}.
\end{itemize}
The equations of motion for an $n$-axis manipulator are given by
\begin{equation}
\vec{Q} =
{\bf M}(\vec{q})\ddot{\vec{q}} + {\bf
C}(\vec{q},\vec{\dot{q}}) \vec{\dot{q}} +
{\bf F}(\vec{\dot{q}}) + {\bf G}(\vec{q}) \label{eq:inv-dyn}
\end{equation}
where

\begin{tabular}{r p{5in}}
$\vec{q}$ & is the vector of generalized joint coordinates describing
the pose of the manipulator\\
$\vec{\dot{q}}$ & is the vector of joint velocities;      \\
$\vec{\ddot{q}}$ & is the vector of joint accelerations   \\
$\bf M$ & is the symmetric joint-space inertia matrix, or manipulator inertia
tensor\\
$\bf C$ & describes Coriolis and centripetal effects --- Centripetal torques are
proportional to $\dot{q}_i^2$, while the Coriolis torques are proportional
to $\dot{q}_i \dot{q}_j$\\
$\bf F$ & describes viscous and Coulomb friction and is not generally considered
part of the rigid-body dynamics\\
$\bf G$ & is the gravity loading\\
$\vec{Q}$ & is the vector of generalized forces associated with the
generalized coordinates $\vec{q}$.
\end{tabular}

The equations may be derived via a number of techniques, including
Lagrangian (energy based), Newton-Euler,
d'Alembert\cite{Fu87,Lee83}\
or Kane's\cite{Kane83} method.
The earliest reported work was by Uicker\cite{Uicker65} and Kahn\cite{Kahn69}\
using the Lagrangian approach.
Due to the enormous computational cost, $O(n^4)$, of this approach it was not
possible to compute manipulator torque for real-time control.
To achieve real-time performance many approaches were suggested, including 
table lookup\cite{Raibert78} and
approximation\cite{Bejczy74,Paul72}.  
The most common
approximation
was to ignore the velocity-dependent term $\bf C$, since
accurate positioning and
high speed motion are exclusive in typical robot applications.

Orin \etal\cite{Orin79} proposed an alternative approach based
on the Newton-Euler (NE) equations of rigid-body motion applied to each link.
Armstrong\cite{Armstrong79} then showed how recursion might
be applied resulting in $O(n)$ complexity.
Luh \etal\cite{Luh80a} provided a recursive formulation of
the
Newton-Euler equations with linear and angular velocities referred to link
coordinate frames.  They suggested a time improvement from $7.9$s for the
Lagrangian formulation to $4.5\unit{ms}$, and thus it became practical to
implement `on-line'.
Hollerbach\cite{Hollerbach80} showed how recursion could be applied to
the Lagrangian form, and reduced the computation to within a factor of 3
of the recursive NE.  Silver\cite{Silver82} showed
the equivalence of the recursive Lagrangian and Newton-Euler forms, and
that the difference in efficiency is due to the representation of angular
velocity.

``Kane's equations'' \cite{Kane83} provide another methodology for
deriving the equations of motion for a specific manipulator.
A number of `Z' variables are introduced, which while not necessarily
of physical significance, lead to a dynamics formulation with low computational
burden.
Wampler\cite{Wampler85} discusses the computational costs of Kane's method
in some detail.

The NE and Lagrange forms can be written generally in terms of the
Denavit-Hartenberg parameters ---
however the specific formulations, such as Kane's, can have lower computational
cost for the specific manipulator.
Whilst the recursive forms are computationally more efficient, the
non-recursive forms compute the individual dynamic terms ($\bf M$, $\bf C$
and $\bf G$) directly.
A comparison of computation costs  is given in Table \ref{tab:dyn-costs}.

\begin{table}
\centering
\begin{tabular}{|l|l|l||r|r|} \hline
Method &        Multiplications & Additions & \multicolumn{2}{c|}{For N=6}
\\ \cline{4-5}
       &                        &           & Multiply & Add \\ \hline
Lagrangian\cite{Hollerbach80} &
	$32\frac{1}{2}n^4 + 86\frac{5}{12}n^3$ &
	$25n^4 + 66\frac{1}{3}n^3$ &
	66,271 & 51,548 \\
    &   $+ 171\frac{1}{4}n^2 + 53\frac{1}{3}n$  &
	$+ 129\frac{1}{2}n^2 + 42\frac{1}{3}n$ & & \\
     &  $-128$ &  $-96$ & & \\
Recursive NE\cite{Hollerbach80} &
	$150n - 48$     & $131n-48$     & 852 & 738     \\
Kane\cite{Kane83} &                  &               & 646 & 394     \\
Simplified RNE\cite{Murray84} &                &               & 224 & 174     \\ \hline
\end{tabular}
\caption[Comparison of computational costs for inverse dynamics.]{Comparison 
of computational costs for inverse dynamics from various sources.  The last
entry is achieved by symbolic simplification using the software package ARM.}
\label{tab:dyn-costs}
\end{table}


\subsection{Recursive Newton-Euler formulation}\label{sec:robot-control}
The recursive Newton-Euler (RNE) formulation\cite{Luh80a}
computes the inverse manipulator dynamics, that is, the joint torques
required for a given set of joint angles, velocities and accelerations.
The forward recursion propagates kinematic information --- such as
angular velocities, angular accelerations, linear accelerations --- from the base
reference frame (inertial frame) to the end-effector.  The backward
recursion propagates the forces and moments exerted on each link from
the end-effector of the manipulator to the base reference frame\footnote{It 
should be noted that using MDH notation with its different
axis assignment conventions the Newton Euler formulation is expressed
differently\cite{Craig89}.}.
Figure \ref{fig:dynamics-notation} shows the variables involved in the
computation for one link.

The notation of Hollerbach\cite{Hollerbach80} and Walker and Orin
\cite{Walker82} will be used in
which the left superscript indicates the reference coordinate
frame for the variable.  The notation of Luh \etal\cite{Luh80a}
and later Lee\cite{Lee82a,Fu87} is considerably less clear.

\begin{figure}
\centerline{\psfig{figure=figs/dyn.id,width=\bigwidth}}
\caption[Notation for inverse dynamics]{Notation used for
inverse dynamics, based on standard Denavit-Hartenberg notation.}
\label{fig:dynamics-notation}
\end{figure}

\paragraph{Outward recursion, $1\leq i\leq n$.}

\begin{eqnarray}
\lefteqn{\mbox{If axis $i+1$ is rotational}}            \nonumber       \\
\mbox{}^{i+1}\vec{\omega}_{i+1} & = &
	{}^{i+1} {\bf R}_i \left( \mbox{}^i\vec{\omega}_i + \vec{z}_0\dot{\vec{q}}_{i+1} \right)	\label{eq:rne-start}
	\\
\mbox{}^{i+1}\dot{\vec{\omega}}_{i+1}  & = &
	{}^{i+1} {\bf R}_i \left\{ \mbox{}^i\dot{\vec{\omega}}_i +
	\vec{z}_0\ddot{\vec{q}}_{i+1} +
	\mbox{}^i\vec{\omega}_i \times \left(\vec{z}_0\dot{\vec{q}}_{i+1}\right)\right\}                    \\
\mbox{}^{i+1}\vec{v}_{i+1} & = &
	\mbox{}^{i+1}\vec{\omega}_{i+1} \times  {}^{i+1}\vec{p}_{i+1}^{*} +
	{}^{i+1} {\bf R}_i \mbox{}^i\vec{v}_i
	\label{eq:v-rot}                        \\
\mbox{}^{i+1}\dot{\vec{v}}_{i+1} & = &
	\mbox{}^{i+1}\dot{\vec{\omega}}_{i+1} \times {}^{i+1}\vec{p}_{i+1}^{*} +
	\mbox{}^{i+1}\vec{\omega}_{i+1} \times \left\{
	    \mbox{}^{i+1}\vec{\omega}_{i+1} \times  {}^{i+1}\vec{p}_{i+1}^{*}
	\right\} + {}^{i+1} {\bf R}_i \mbox{}^i\dot{\vec{v}}_i
	\\
\lefteqn{\mbox{If axis $i+1$ is translational}}         \nonumber       \\
\mbox{}^{i+1}\vec{\omega}_{i+1} & = &
	{}^{i+1} {\bf R}_i \mbox{}^i\vec{\omega}_i
	\\
\mbox{}^{i+1}\dot{\vec{\omega}}_{i+1}  & = &
	{}^{i+1} {\bf R}_i  \mbox{}^i\dot{\vec{\omega}}_i
	\\
\mbox{}^{i+1}\vec{v}_{i+1} & = &
	{}^{i+1} {\bf R}_i \left(\vec{z}_0\dot{\vec{q}}_{i+1} + \mbox{}^i\vec{v}_i\right) +
	\mbox{}^{i+1}\vec{\omega}_{i+1} \times {}^{i+1}\vec{p}_{i+1}^{*}
	\label{eq:v-trans}                       \\
\mbox{}^{i+1}\dot{\vec{v}}_{i+1} & = &
	{}^{i+1} {\bf R}_i \left(\vec{z}_0 \ddot{\vec{q}}_{i+1} + \mbox{}^i\dot{\vec{v}}_i\right) +
	\mbox{}^{i+1}\dot{\vec{\omega}}_{i+1} \times {}^{i+1}\vec{p}_{i+1}^{*} +
	2\; \mbox{}^{i+1}\vec{\omega}_{i+1}  \times \left({}^{i+1} {\bf R}_i \vec{z}_0
	\dot{\vec{q}}_{i+1}\right) \nonumber \\
	&& + \mbox{}^{i+1}\vec{\omega}_{i+1} \times \left(
		\mbox{}^{i+1}\vec{\omega}_{i+1} \times {}^{i+1}\vec{p}_{i+1}^{*}\right)
\end{eqnarray}
\begin{eqnarray}
\mbox{}^i\dot{\vec{\overline{v}}}_i & = &
	\mbox{}^i\dot{\vec{\omega}}_i \times \vec{s}_i +
	\mbox{}^i\vec{\omega}_i \times \left\{
		\mbox{}^i\vec{\omega}_i \times \vec{s}_i
	\right\} +
	\mbox{}^i\dot{\vec{v}}_i \label{eq:rne-vbar}\\
\mbox{}^i\vec{F}_i & = & m_i \mbox{}^i\dot{\vec{\overline{v}}}_i \label{eq:rne-Fi}\\
\mbox{}^i\vec{N}_i & = &
	{\bf J}_i  \mbox{}^i\dot{\vec{\omega}}_i +
	\mbox{}^i\vec{\omega}_i \times \left({\bf J}_i \mbox{}^i\vec{\omega}_i\right)
\end{eqnarray}

\paragraph{Inward recursion, $n \geq i \geq 1$.}
\begin{eqnarray}
\mbox{}^i\vec{f}_i  & = &
	{}^i {\bf R}_{i+1}\,  \mbox{}^{i+1}\vec{f}_{i+1} + \mbox{}^i\vec{F}_i\\
\mbox{}^i\vec{n}_i & = &
	{}^i {\bf R}_{i+1} \, \left\{ \mbox{}^{i+1}\vec{n}_{i+1} +
	\left( {}^{i+1} {\bf R}_i \mbox{}^i\vec{p}_i^{*} \right) \times {}^i \mbox{}^{i+1}\vec{f}_{i+1}\right\} +
	\left(\mbox{}^i\vec{p}_i^{*} + \vec{s}_i\right) \times \mbox{}^i\vec{F}_i +
	\mbox{}^i\vec{N}_i  \label{eq:rne-ni} \\
\vec{Q}_i & = &  \left\{ \begin{array}{ll}
	\left(\mbox{}^i\vec{n}_i\right)^T \left({}^i {\bf R}_{i+1} \vec{z}_0\right) &
		\mbox{if link $i+1$ is rotational} \\
	\left(\mbox{}^i\vec{f}_i\right)^T \left({}^i {\bf R}_{i+1} \vec{z}_0\right)&
		\mbox{if link $i+1$ is translational}
	\end{array} \right.				\label{eq:rne-end}
\end{eqnarray}
where

\begin{longtable}{r p{12cm}}
$i$ & is the link index, in the range $1$ to $n$ \\
${\bf J}_i$ & is the moment of inertia of link $i$ about its COM\\
$\vec{s}_i$ & is the position vector of the COM of link $i$ with respect to
frame $i$\\
$\vec{\omega}_i$ & is the angular velocity of link $i$\\
$\dot{\vec{\omega}}_i$ & is the angular acceleration of link $i$\\
$\vec{v}_i$ & is the linear velocity of frame $i$\\
$\dot{\vec{v}}_i$ & is the linear acceleration of frame $i$\\
$\vec{\overline{v}}_i$ & is the linear velocity of the COM of link $i$ \\
$\dot{\vec{\overline{v}}}_i$ & is the linear acceleration of the COM of link
$i$ \\
$\vec{n}_i$ & is the moment exerted on link $i$ by link $i-1$ \\
$\vec{f}_i$ & is the force exerted on link $i$ by link $i-1$ \\
$\vec{N}_i$ & is the total moment at the COM of link $i$\\
$\vec{F}_i$ & is the total force at the COM of link $i$ \\
$\vec{Q}_i$ & is the force or torque exerted by the actuator at joint $i$\\
${}^{i-1} {\bf R}_i$ & is the orthonormal rotation matrix defining
frame $i$ orientation with respect to frame $i-1$.  It is the upper $3 \times
3$
portion of the link transform matrix given in (\ref{eq:Amatrix}).
\begin{eqnarray}
{}^{i-1} {\bf R}_i &=& \left[ \begin{array}{ccc}
 \cos \theta_i & -\cos \alpha_i \sin \theta_i & \sin \alpha_i \sin \theta_i \\
 \sin \theta_i & \cos \alpha_i \cos \theta_i & -\sin \alpha_i \cos \theta_i \\
 0             & \sin \alpha_i               & \cos \alpha_i
 \end{array} \right] \\
{}^{i} {\bf R}_{i-1} &=& ( {}^{i-1} {\bf R}_{i} ) ^{-1} = ( {}^{i-1} {\bf R}_{i} ) ^T
\end{eqnarray}
\\
${}^i \vec{p}_i^{*}$ & is the displacement from the origin of frame $i-1$
to frame $i$ with respect to frame $i$.
\begin{equation}
\mbox{}^i\vec{p}_i^{*} = \left[ \begin{array}{c}
	a_i \\
	d_i \sin \alpha_i \\
	d_i \cos \alpha_i
	\end{array} \right]
\end{equation}
It is the negative translational part of $({}^{i-1} {\bf A}_{i})^{-1}$.\\
$\vec{z}_0$ & is a unit vector in Z direction, $\vec{z}_0 = [0\;0\;1]$ \\
\end{longtable}

\noindent
Note that the COM linear velocity given by equation (\ref{eq:v-rot}) or
(\ref{eq:v-trans}) does not need to be computed since no other
expression depends upon it.
Boundary conditions are used to introduce the effect of gravity by setting the
acceleration of the base link
\be
\dot{v}_0 = -\vec{g}
\ee
where $\vec{g}$ is the gravity vector in the reference coordinate 
frame, generally acting in the negative Z direction, downward.
Base velocity is generally zero
\begin{eqnarray}
v_0 &=& 0       \\
\omega_0 &=& 0 \\
\dot{\omega}_0 &=& 0
\end{eqnarray}

At this stage the Toolbox only provides an implementation of this algorithm
using the standard Denavit-Hartenberg conventions.

\subsection{Direct dynamics}  \label{robot:direct-dyn}
Equation (\ref{eq:inv-dyn}) may be used to compute the so-called inverse dynamics, that is,
actuator torque as a function
of manipulator state and is useful for on-line control.
For simulation the direct, integral or \DEF{forward dynamic} formulation is required 
giving joint motion in terms of input torques.  

Walker and Orin\cite{Walker82} describe several
methods for computing the forward dynamics, and all make use of an
existing inverse dynamics solution.
Using the RNE algorithm for inverse dynamics, the computational complexity of
the forward dynamics using `Method 1'  is $O(n^3)$ for an n-axis manipulator.
Their other methods are increasingly more sophisticated but reduce the computational
cost, though still $O(n^3)$.
Featherstone\cite{Featherstone87}\
has described the ``articulated-body method'' for $O(n)$ computation of
forward dynamics, however for $n < 9$ it is more expensive than the 
approach of Walker and Orin.
Another $O(n)$ approach for forward
dynamics has been described by Lathrop\cite{Lathrop86}.

\subsection{Rigid-body inertial parameters} \label{sec:robot-rigidbody-params}
Accurate model-based dynamic control of a manipulator requires knowledge of the
rigid-body inertial parameters.
Each link has ten independent  inertial parameters:
\begin{itemize}
\item link mass, $m_i$;
\item three first moments, which may be expressed
as the COM location, $\vec{s}_i$, with respect to some datum on the link
or as a moment $\vec{S}_i = m_i \vec{s}_i$;
\item six second moments, which represent the inertia of the link about a
given axis, typically through the COM.
The second moments may be expressed in matrix or tensor form as
\be
{\bf J} = \left[ \begin{array}{ccc}
		J_{xx} & J_{xy} & J_{xz}\\
		J_{xy} & J_{yy} & J_{yz}\\
		J_{xz} & J_{yz} & J_{zz}
	   \end{array}
	\right]		\label{eq:inertia-matrix}
\ee
where the diagonal elements are the \DEF{moments of inertia}, and the
off-diagonals are \DEF{products of inertia}.
Only six of these nine elements are unique:
three moments and three products of inertia.

For any point in a rigid-body there is one set of axes known as the
\DEF{principal axes of inertia} for which the off-diagonal terms, or products, are
zero.  These axes are given by the eigenvectors of the
inertia matrix \eq{eq:inertia-matrix} and the eigenvalues are the principal moments of
inertia.  Frequently the products of inertia of the robot links are zero due 
to symmetry.
\end{itemize}

A 6-axis manipulator rigid-body dynamic model thus entails 60 
inertial parameters.
There may be additional parameters per joint due to friction and motor 
armature inertia.  
Clearly, establishing numeric values for this number of parameters is a difficult task.
Many parameters cannot be measured without dismantling the robot and performing
careful experiments, though this approach was used by Armstrong \etal\cite{Armstrong86}.
Most parameters could be derived from CAD models of the robots, but this information
is often considered proprietary and not made available to researchers.

\bibliographystyle{ieeetr}
\bibliography{strings,robot,control,dynamics,kinematics,force,grind,publist,software}
\end{document}
